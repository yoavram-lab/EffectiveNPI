\documentclass[12pt]{extarticle}
\usepackage{geometry}
\geometry{
a4paper,
total={170mm,257mm},
left=20mm,
top=20mm,
headheight=12pt
}

\usepackage[parfill]{parskip} % Activate to begin paragraphs with an empty line rather than an indent
\usepackage{graphicx} % Use pdf, png, jpg, or eps§ with pdflatex; use eps in DVI mode
% TeX will automatically convert eps --> pdf in pdflatex		

\usepackage{amssymb,amsmath,amsthm}
\usepackage{commath}
\usepackage{longtable}
\usepackage[hyphens]{url}
\usepackage[dvipsnames]{xcolor}
\usepackage[unicode=true,colorlinks=true,urlcolor=Blue,citecolor=black,linkcolor=black]{hyperref}
\def\equationautorefname~#1\null{Eq.~#1\null}
\PassOptionsToPackage{hyphens}{url} % url is loaded by hyperref
\usepackage{authblk}
\usepackage{lipsum}
\usepackage{multicol}
\usepackage{titlesec}	
\usepackage[font=small,labelfont=bf]{caption}
\usepackage{enumitem}
\usepackage{soul}
\usepackage{booktabs}
\usepackage{pgfplotstable} % http://mirrors.ctan.org/graphics/pgf/contrib/pgfplots/doc/pgfplotstable.pdf
\usepackage{subcaption}
\captionsetup[subfigure]{labelformat=empty}
\usepackage{dcolumn}
\newcolumntype{d}[1]{D{.}{.}{#1}}
\newcommand\ccell[1]{\multicolumn{1}{c}{#1}} % a wild guess...
\usepackage{adjustbox}
\usepackage{lscape}

%SetFonts
% newtxtext+newtxmath
\usepackage{newtxtext} %loads helv for ss, txtt for tt
\usepackage{amsmath}
\usepackage[bigdelims]{newtxmath}
\usepackage[T1]{fontenc}
\usepackage{textcomp}
%SetFonts

\renewcommand{\thefootnote}{\fnsymbol{footnote}}

% less space before sections 
% \@startsection {NAME}{LEVEL}{INDENT}{BEFORESKIP}{AFTERSKIP}{STYLE} 
%            optional * [ALTHEADING]{HEADING} 
\makeatletter
 \renewcommand\section{\@startsection {section}{1}{\z@}%
     {-2.5ex \@plus -1ex \@minus -.2ex}%
     {1.3ex \@plus.2ex}%
    {\Large\bfseries}}
    
% Species names
%% Meta-Command for defining new species macros
\usepackage{xspace}

\newcommand{\species}[3]{%
  \newcommand{#1}{\gdef#1{\textit{#3}\xspace}\textit{#2}\xspace}}
  
\species{\yeast}{Saccharomyces cerevisiae}{S.~cerevisiae}
\species{\calbicans}{Candida albicans}{C.~albicans}
\species{\cneoformans}{Cryptococcus neoformans}{C.~neoformans}

% COVID-19
\newcommand{\covid}{COVID\=/19 }
\newcommand{\sars}{SARS\=/CoV\=/2 }

% line numbers
 \usepackage[displaymath, mathlines]{lineno}
 \renewcommand\linenumberfont{\normalfont\small\sffamily}
\linenumbers
\modulolinenumbers[2]

% Yoav & Lee commands
\newcommand*{\tr}{^\intercal}
\let\vec\mathbf
\newcommand{\matrx}[1]{{\left[ \stackrel{}{#1}\right]}}
\newcommand{\diag}[1]{\mbox{diag}\matrx{#1}}
\newcommand{\goesto}{\rightarrow}
\newcommand{\dspfrac}[2]{\frac{\displaystyle #1}{\displaystyle #2} }
\newtheorem{theorem}{Theorem}
\newtheorem{corollary}{Corollary}
\newtheorem{lemma}{Lemma}
\newtheorem{remark}{Remark}
\newtheorem{result}{Result}
\renewcommand\qedsymbol{} % no square at end of proof
\newcommand{\cl}{\mathbf{L}}
\newcommand{\cj}{\mathbf{J}}
\newcommand{\ci}{I}


% Supplementary
\newcommand{\beginsupplement}{%
      	\setcounter{table}{0}
        \renewcommand{\thetable}{S\arabic{table}}%
        \setcounter{figure}{0}
        \renewcommand{\thefigure}{S\arabic{figure}}%
}
     
% NatBib
\usepackage[super,comma]{natbib}
%\usepackage[numbers,super,sort&compress]{natbib}
%\usepackage[round,colon,authoryear]{natbib}
\renewcommand{\bibsection}{}
%\renewcommand{\bibfont}{\small}

% unbreakable dashes
\usepackage[shortcuts]{extdash}

% Title page
\title{Inferring the effective start dates of non-pharmaceutical interventions during \covid outbreaks}

% Authors
\renewcommand\Affilfont{\small}

\author[1]{Ilia Kohanovski}
\author[2,3]{Uri Obolski}
\author[1,4,a]{Yoav Ram}

\affil[1]{School of Computer Science, Interdisciplinary Center Herzliya, Herzliya 4610101, Israel}
\affil[2]{School of Public Health, Faculty of Medicine, Tel Aviv University, Tel Aviv 6997801, Israel}{
\affil[3]{Porter School of the Environment and Earth Sciences, Faculty of Exact Sciences, Tel Aviv University, Tel Aviv 6997801, Israel}
\affil[4]{School of Zoology, Faculty of Life Sciences, Tel Aviv University, Tel Aviv 6997801, Israel}
\affil[a]{Corresponding author: yoav@yoavram.com, ORCID 0000-0002-9653-4458}

% Document
\begin{document}
\maketitle


%\pagebreak 

% Abstract
\begin{abstract}
During Feb-Apr 2020, many countries implemented non-pharmaceutical interventions, such as school closures and lockdowns, with variable schedules, to control the \covid pandemic caused by the \sars virus.
Overall, these interventions seem to have reduced the spread of the pandemic.
We hypothesize that the official and effective start date of such interventions can be noticeably different, for example due to slow adoption by the population, or because the authorities and the public are unprepared.
We fit an SEIR model to case data from~12 regions to infer the effective start dates of interventions and contrast them with the official dates.
We find mostly late, but also early effects of interventions. For example, Italy implemented a nationwide lockdown on Mar~11, but we infer the effective date on Mar~17 ($\substack{+3.05 \\ -2.01}$~days 95\% CI).
In contrast, Germany announced a lockdown on Mar~22, but we infer an effective start date on Mar~19 ($\substack{+0.92 \\ -0.99}$~days 95\% CI).
We demonstrate that differences between the official and effective start of NPIs can lead to under-estimating their impact, and discuss potential causes and consequences of our results.
\\\\
{\small
Keywords: SEIR, COVID-19, public health, epidemic, infectious disease, NPI
}
\end{abstract}


\pagebreak


% Table - NPI dates %%%
\begin{table}[b!]
\centering
\pgfplotstabletypeset[
    col sep=comma,
    string type,
    every head row/.style={before row=\hline,after row=\hline},
    every last row/.style={after row=\hline},
    ]{../data/NPI_dates.csv}
\caption{
\textbf{Official start of non-pharmaceutical interventions.}
The date of the first intervention is for a ban of public events, or encouragement of social distancing, or for school closures.
In all countries except Sweden, the date of the last intervention ($\tau^*$) is for a lockdown. In Sweden, where a lockdown was not ordered during the studied dates, the last date is for school closures. Dates for European countries from \citet{Flaxman2020}, date for Wuhan, China from \citet{Pei2020}. See \autoref{fig:NPI_dates} for a visual presentation.
}
\label{table:NPI_dates}
\end{table}



% Introduction
\section*{Introduction}

The \covid pandemic has resulted in implementation of extreme non-pharmaceutical interventions (NPIs) in many affected countries. These interventions, from social distancing to lockdowns, are applied in a rapid and widespread fashion.
NPIs are designed and assessed using epidemiological models, which follow the dynamics of infection to forecast the effect of different mitigation and suppression strategies on the levels of infection, hospitalization, and fatality.
These epidemiological models usually assume that the effect of NPIs on infection dynamics begins at the officially declared date~\citep{Flaxman2020,Gatto2020,Li2020}.

Adoption of public-health recommendations is often critical for effective response to infectious diseases, and has been studied in the context of HIV~\citep{Kaufman2014} and vaccination~\citep{Dunn2015,Wiyeh2018}, for example.
However, behavioural and social change does not occur immediately, but rather requires time to diffuse in the population through media, social networks, and social interactions. 
Moreover, compliance to NPIs may differ between different interventions and between people with different backgrounds.
For example, in a survey of 2,108 adults in the UK during Mar~2020, \citet{Atchison2020} found that those over 70 years old were more likely to adopt social distancing than young adults (18-34 years old), and that those with lower income were less likely to be able to work from home and to self-isolate.
Similarly, compliance to NPIs may be impacted by personal experiences. \citet{Smith2020} have surveyed 6,149 UK adults in late Apr~2020 and found that people who believe they have already had \covid are more likely to think they are immune, and less likely to comply with social distancing guidelines. 
Compliance may also depend on risk perception as perceived by the the number of domestic cases or even by reported cases in other regions and countries.
Interestingly, the perceived risk of \covid infection has likely caused a reduction in the number of influenza-like illness cases in the US starting from mid-Feb~2020~\citep{Zipfel2020}.

Here, we hypothesise that there is a significant difference between the official start of NPIs and their effective adoption by the public and therefore their effect on infection dynamics.
We use a \textit{Susceptible-Exposed-Infected-Recovered} (SEIR) model and a \textit{Markov Chain Monte Carlo} (MCMC) parameter estimation framework to infer the effective start date of NPIs from publicly available \covid case data in 12 geographical regions.
We compare these estimates to the official dates, and find that they include both late and early effects of NPIs on infection dynamics.
We conclude by demonstrating how differences between the official and effective start of NPIs can confound assessments of their impacts.



% Results
\section*{Results}


%%% Fig - tau summary %%%
% generated with /python Fig_tau_summary.py FOLDER NAME
\begin{figure}[b!]
    \centering
	\includegraphics[width=0.6\textwidth]{../figures/Fig-summary-Apr11.pdf}	
    \caption{
    \textbf{Official vs. effective start of non-pharmaceutical interventions.}
    	The difference between $\tau$ the effective and $\tau^*$ the official start of NPIs is shown for different regions. The effective date is delayed in UK, Austria, Italy, France, Belgium, Spain, and Wuhan, China, compared to the official date (red markers). In contrast, the estimated effective dates in Sweden, Denmark, and Germany are earlier than the official dates (blue markers), although uncertainty is low only for Germany (i.e., zero is not in 95\% CI). The credible intervals for Sweden, Denmark, and Norway are especially wide, see text and \autoref{fig:early} for possible explanation.
	Here, the markers show $\hat{\tau}$, the marginal posterior median (\autoref{table:estimated-params}). $\tau^*$ is the last NPI date (a lockdown, except in Sweden; \autoref{table:NPI_dates}). Thin and bold lines show 95\% and 75\% credible intervals, respectively. \autoref{fig:tau-summary-mar28} shows a similar summary when estimating $\hat{\tau}$ using case data up to Mar~28 rather than Apr~11.
    }
    \label{fig:tau-summary}
\end{figure}



Several studies have described the effects of non-pharmaceutical interventions in different geographical regions~\citep{Flaxman2020,Gatto2020,Li2020}. 
Some of these studies have assumed that the parameters of the epidemiological model change at a specific date (\autoref{eq:NPI_model}), and set the change date $\tau$ to the official NPI date $\tau^*$, usually the lockdown start date (\autoref{table:NPI_dates}).
They then fit the model once for time $t<\tau^*$ and once for time $t \ge \tau^*$.
For example, \citet{Li2020} estimate the infection dynamics in China before and after $\tau^*$, which is set at Jan 23, 2020. Thereby, they effectively estimate the transmission and reporting rates before and after $\tau^*$ separately.

Here, we estimate the joint posterior distribution of the effective start date of NPIs, $\tau$, and the transmission and reporting rates before and after $\tau$ from the entire data, rather than splitting the data at~$\tau$. 
This is done under the simplifying assumption that all interventions start at a specific date, despite the reality that the durations between the first and last NPIs were between 4 and 12 days (\autoref{table:NPI_dates}, \autoref{fig:NPI_dates}).
We then estimate $\hat{\tau}$ as the median of the marginal posterior distribution of $\tau$.
Credible intervals (CI) are calculated as the highest density intervals~\citep{Kruschke2014}, and their upper and lower boundaries are reported as $\substack{+\text{upper} \\ -\text{lower}}$ in days relative to $\hat{\tau}$.

We compare the posterior predictive plots of a model with  free~$\tau$ with those of a model with~$\tau$ fixed at~$\tau^*$ and a model without~$\tau$ (i.e. transmission and reporting rates are constant).
All three models were fitted to case data up to Apr~11, used to predict out-of-sample case data up to Apr~24, and these predictions were then compared to the real case data. 
The model with free~$\tau$ clearly produces better and less variable predictions~(\autoref{fig:ppc}):
In all 11 european countries, the expected posterior RMSE (root mean squared error) of the out-of-sample predictions is lowest for the model with a free~$\tau$ (\autoref{table:RMSE}).
When we compare the models using WAIC~(\autoref{eq:WAIC}), the model with free~$\tau$ is strongly preferred in 9 out of 12 countries, the exceptions being Norway (only marginal preference), Sweden, and Denmark (\autoref{table:WAIC}).
Indeed, we estimate low effect of NPIs on transmission in Denmark and Sweden (i.e. $\lambda=0.7$ and $0.74$, respectively; see~\autoref{table:estimated-params}).
This may interfere with the inference of $\tau$ due to unidentifiability. 
Notably, the data for Sweden and Denmark do not have a single "peak" during the evaluated dates, possibly leading to wide credible intervals on $\tau$ (\autoref{fig:tau-summary}) and poor WAIC in the model with free~$\tau$, whereas the duration between the first and last interventions was especially long in Norway (\autoref{table:NPI_dates}).

We compare the official $\tau^*$ and effective $\tau$ start of NPIs and find that in most regions the effective start of NPIs differs from the official date: in 10 of 12 countries the 75\% credible interval on $\tau$ does not include $\tau^*$ (7 of 12 countries when considering a 95\% interval; \autoref{fig:tau-summary}).
The exceptions are Norway and Switzerland (see below).
The former also has a wide credible interval, perhaps because it has the longest duration between the first and last NPIs~(\autoref{table:NPI_dates}).



%%% Fig - Late %%%
\begin{figure}[h]
    \centering
    \begin{subfigure}{0.45\textwidth}
        \includegraphics[width=\textwidth]{../figures/posterior/Apr11/Italy_τ_posterior.pdf}
    \end{subfigure}
  	~
    \begin{subfigure}{0.45\textwidth}
        \includegraphics[width=\textwidth]{../figures/posterior/Mar28/Wuhan_τ_posterior.pdf}
    \end{subfigure}
    \\
    \begin{subfigure}{0.45\textwidth}
        \includegraphics[width=\textwidth]{../figures/posterior/Apr11/Spain_τ_posterior.pdf}
    \end{subfigure}
    ~
    \begin{subfigure}{0.45\textwidth}
		\includegraphics[width=\textwidth]{../figures/posterior/Apr11/France_τ_posterior.pdf}
    \end{subfigure}
    \caption{
	\textbf{Late effect of non-pharmaceutical interventions.}
    Posterior distribution of $\tau$, the effective start date of NPI, is shown as a histogram of MCMC samples. Red line shows the official last NPI date $\tau^*$. Black line shows the estimated effective start date $\hat{\tau}$. Shaded area shows a 95\% credible interval. 
    }
    \label{fig:late}
\end{figure}



%%% Late
\paragraph*{Late effective start of NPIs.}
In half of the examined regions, we estimate that the effective start of NPIs $\tau$ is later than the official date $\tau^*$.
 
In Italy, the first case was officially confirmed on Feb~21. School closures were implemented on Mar~5~\citep{Flaxman2020}, a lockdown was declared in Northern Italy on Mar~8, with social distancing implemented in the rest of the country, and the lockdown was extended to the entire nation on Mar 11~\citep{Gatto2020}.
That is, the first and last official NPI dates are Mar~5 and Mar~11.
However, we estimate the effective date ($\hat{\tau}$) six~days after the lockdown, at Mar~17 ($\substack{+3.05 \\ -2.01}$~days 95\% CI; \autoref{fig:late}). 

In Wuhan, China, a lockdown was ordered on Jan 23~\citep{Li2020}, but we estimate the effective start of NPIs to be ten days later, at Feb~2 ($\substack{+2.29 \\ -2.97}$~days 95\% CI). Yet, there is low but noticeable posterior probability for Jan~25 (\autoref{fig:late}), for which the effect of NPIs on transmission is considerably lower (\autoref{fig:joint}).

In Spain, social distancing was encouraged starting on Mar~8~\citep{Flaxman2020}, but mass gatherings still occurred on Mar~8, including a march of 120,000 people for the \href{https://www.nytimes.com/2020/03/13/world/europe/spain-coronavirus-emergency.html}{International Women's Day}, and a football match between \href{https://www.espn.com/soccer/match?gameId=550350}{Real Betis and Real Madrid} (final score 2--1) with a crowd of 50,965 in Seville.
A national lockdown was only announced on Mar~14~\citep{Flaxman2020}.
Nevertheless, we estimate the effective start of NPI eleven days later, at Mar~25($\substack{+1.70 \\ -1.43}$~days 95\% CI, \autoref{fig:late}).

Similarly, in France we estimate the effective start of NPIs at Mar~24 ($\substack{+2.65 \\ -1.44}$~days 95\% CI, \autoref{fig:late}).
This is a week later than the official lockdown, which started at Mar~17, and more than 10 days after the earliest NPI, banning of public events, which started on Mar~13~\citep{Flaxman2020}.



%%% Fig - Early %%% 
\begin{figure}[h]
    \centering
    \begin{subfigure}{0.45\textwidth}
        \includegraphics[width=\textwidth]{../figures/posterior/Apr11/Germany_τ_posterior.pdf}
    \end{subfigure}
    ~
    \begin{subfigure}{0.45\textwidth}
		\includegraphics[width=\textwidth]{../figures/posterior/Apr11/Sweden_τ_posterior.pdf}
    \end{subfigure}
    
	\begin{subfigure}{0.45\textwidth}
        \includegraphics[width=\textwidth]{../figures/posterior/Apr11/Switzerland_τ_posterior.pdf}
    \end{subfigure}
    ~
    \begin{subfigure}{0.45\textwidth}
		\includegraphics[width=\textwidth]{../figures/posterior/Apr11/Norway_τ_posterior.pdf}
    \end{subfigure}
    \caption{
    \textbf{Early and exact effect of non-pharmaceutical interventions.}
    Posterior distribution of $\tau$, the effective start date of NPI, is shown as a histogram of MCMC samples. Red line shows the official last NPI date $\tau^*$. Black line shows the estimated effective start date $\hat{\tau}$. Shaded area shows a 95\% credible interval. 
	}
	\label{fig:early}
\end{figure}



%%% Early
\paragraph*{Early effective start of NPIs.}
In some regions we estimate an effective start of NPIs $\tau$ that is \emph{earlier} then the official date $\tau^*$ (\autoref{fig:tau-summary}).
The only conclusive early case is Germany, in which we estimate the effective start of NPIs at Mar~19 ($\substack{+0.92 \\ -0.99}$~days 95 \%CI; \autoref{fig:early}).
This estimate falls between the first and last official NPI dates, Mar~12 and Mar~22 (\autoref{table:NPI_dates}). Therefore, when we refer to this case as "early", we mean that the effective date (Mar~19) occurs \emph{before} the official lockdown date (Mar~22), not that it occurs before all NPIs.
Interestingly, Germany had the second longest duration between first and last NPIs after Norway (10 and 12 days respectively; \autoref{table:NPI_dates}).
However, the credible interval for the effective start date in Germany is narrow (1.91 days), whereas it is very wide in Norway (17.61 days).
The significantly earlier estimate of $\hat{\tau}$ relative to $\tau^*$ can suggest that early NPIs in Germany more effectively reduced transmission rates compared to other countries. Another possible interpretation is that the German population anticipated the lockdown and reacted before it started.

We also estimate an early effective start of NPIs in Denmark, Norway, and Sweden.
However, the credible intervals are quite wide (\autoref{fig:tau-summary}), and in Denmark and Sweden the evidence did not support the model with free~$\tau$ over the model with $\tau$ fixed at the official date (WAIC values in \autoref{table:WAIC}).
Indeed, Denmark and Sweden had the smallest estimated effect of NPIs on transmission rates, which  probably hinders our ability to estimate $\tau$ in these countries (\autoref{table:estimated-params}).
Moreover, in Sweden the number of daily cases continued to grow up to Apr~11, rather than "peak" (\autoref{fig:ppc}). 
In Denmark, the opposite occurred: there were seemingly two "peaks", on Mar~11 and on Apr~8 (\autoref{fig:ppc}); the first "peak" may be a result of stochastic events, for example due to a large cluster of cases or an accumulation of tests.
We suspect that these missing and additional "peaks" increase the uncertainty in our inference.

The estimated effective date in Norway is Mar~22, two days earlier than the official date of Mar~24.
However, the posterior distribution is very wide ($\substack{+5.59 \\ -12.03}$~days 95\% CI): it covers the range between Mar~10, two days before the first NPI, and Mar~27, three days after the last NPI (\autoref{table:NPI_dates}, \autoref{fig:NPI_dates}). 
The high uncertainty might be due to the long duration between the first and last NPIs; however, Germany had the second longest duration between first and last NPIs, and the corresponding posterior distribution is very narrow (\autoref{fig:early}).
There may also be an unidentifiability issues between $\tau$ and the effect of NPIs on transmission (\autoref{fig:joint}). 



%%% Exact
\paragraph*{Exact effective start of NPIs.}
We find one case in which the official and effective dates match and the credible interval is narrow.
Switzerland ordered a national lockdown on Mar~20, after banning public evens and closing schools on Mar~13 and~14~\citep{Flaxman2020}.
Indeed, the posterior median $\hat{\tau}$ is Mar~19 ($\substack{+3.2 \\ -1.78}$~days 95\% CI, see \autoref{fig:early}). It's also worth mentioning that Switzerland was the first to mandate self isolation of confirmed cases~\citep{Flaxman2020}.



%%% Assessment
\paragraph*{Assessment of impact of NPIs.}

The \emph{effective reproduction number} $R$ is the average number of secondary
cases caused by an infected individual after an epidemic is already underway \citep{Bar-On2020a}.
We infer model-based effective reproduction numbers before and after the implementation of NPIs from model parameters (\autoref{eq:Re}).
We then estimate the impact of NPIs as the relative reduction in the reproduction number \citep{Flaxman2020}, $\Delta R = (R_1-R_2)/R_1$. 
We compare the impacts estimated using the fixed~$\tau$ model and the free~$\tau$ model.
That is, we compare the impact estimate assuming that NPIs started at the official date $\tau^*$, versus the estimate when inferring the effective start of NPIs from the data.
\autoref{fig:Re} demonstrates that estimates from the fixed~$\tau$ model (y-axis) are consistently lower than estimates from the free~$\tau$ model (x-axis), except in Sweden and Denmark, in which estimation uncertainty is high.
That is, assuming NPIs start at their official date leads to error in the estimation of their impact.
Moreover, these errors increase with the duration of the delay between the official and effective dates (\autoref{fig:Re2}).
These results suggest that the impact of past NPIs is likely to be under-estimated by health officials and researchers if they assume NPIs effectively start at their official dates.
The estimated impact can then be interpreted as ineffectiveness of the NPIs, leading to more aggressive NPIs being applied.



%%% Fig - Re
% generated with
% > source ./Re.sh
% > python Fig_Re.py
\begin{figure}[h]
    \centering
	\includegraphics[width=\textwidth]{../figures/Fig_Re.pdf}
    \caption{
    \textbf{Impact of NPIs is under-estimated when assuming they start at the official date.}
    Shown are estimates of the relative reduction in the effective reproduction number ($R$, see \autoref{eq:Re}), which measures the impact of NPIs on disease transmission, in 11 European countries.
     The y-axis shows estimates when assuming the start of NPIs is fixed at the official date (fixed~$\tau$); the x-axis shows estimates when inferring the effective start of NPIs from the data (free $\tau$). The dashed line shows a one-to-one correspondence.
     Markers and bars denote the posterior median and 50\% credible intervals (HDI).
    The relative reductions in $R$ are consistently lower for the fixed~$\tau$ model (below the dashed line), except in Sweden and Denmark in which uncertainty is high.
    } 
    \label{fig:Re}
\end{figure}



\pagebreak
% Discussion
\section*{Discussion}

We have inferred the effective start date of NPIs in several geographical regions using an SEIR model under an MCMC parameter estimation framework.
We find examples of both late and early effective start of NPIs relative to the official date (\autoref{fig:tau-summary}).

In most investigated regions we find late effective start of NPIs.
For example, in Italy and in Wuhan, China, the effective start of the lockdowns seems to have occurred five or more days after the official date (\autoref{fig:late}).
This difference might be explained, in some cases, by low compliance or non-adherence to guidelines: In Italy, for example, the government plan to implement a lockdown in the Northern provinces leaked to the public, resulting in people leaving these provinces before the lockdown started~\citep{Gatto2020}.
Late effect of NPIs may also be due to the time required by both the government and the citizens to prepare for a lockdown, and for new guidelines to be adopted by the population.
 
In contrast, in some regions we inferred reduced transmission rates even before official lockdowns were implemented, although this is only conclusive in Germany (\autoref{fig:early}).
An early effective date might be due to early adoption of social distancing and similar behavioural adaptations in parts of the population, possibly due to earlier NPIs or NPIs being applied in other regions.
Adoption of these behaviours may occur via media and social networks, rather than official guidelines, and may be influenced by increased risk perception due to domestic or international \covid reports~\citep{Arthur2020}.
Indeed, the evidence supports a change in infection dynamics (i.e. a model with fixed or free $\tau$) even for Sweden~(\autoref{table:WAIC}, \autoref{table:RMSE}, \autoref{fig:ppc}), where a lockdown was not implemented\footnote{Sweden banned public events on Mar~12, encouraged social distancing on Mar~16, and closed schools on Mar~18~\citep{Flaxman2020}.}.

Interestingly, the effective start of NPIs in France and Spain is estimated to have started on Mar~24 and~25, respectively,
although the official NPI dates differ significantly: the first NPI in France is only one day before the last NPI in Spain.
The number of daily cases was similar in both countries until Mar~8, but diverged by Mar~13, reaching much higher numbers in Spain (\autoref{fig:fig-fr-vs-es}).
This may suggest correlations between effective starts of NPIs due to global or international events.

As expected, we have found that the evidence supports a model in which the transmission rate changes at a specific time point over a model with a constant transmission rate (Tables~\ref{table:WAIC} and~\ref{table:RMSE}).
It may be interesting to check if the evidence supports a model with \emph{two} or more change-points, rather than one. 
Multiple change-points could reflect escalating NPIs (e.g. school closures followed by lockdowns), or an intervention followed by a relaxation.
However, interpretation of such models will be harder, as multiple change-points are also likely to result in parameter unidentifiability, for example due to simultaneous implementations of NPIs~\citep{Flaxman2020}.

As different countries experiment with various intervention strategies, we expect similar shifts to occur: in some cases the population will be late to comply with new guidelines, whereas in other cases the population will adopt either restrictions or relaxations even before they are formally announced.
Attempts to assess the impact of NPIs~\citep{Banholzer2020,Flaxman2020} generally assume they start at their official date, and that a significant change in the dynamics can be observed roughly seven days after the start of NPIs (due to the characteristic serial interval of \covid\citep{Ali2020}).
However, late and early effective start of NPIs, such as we have inferred, can bias these assessments and lead to wrong conclusions about the impact of NPIs (\autoref{fig:Re}).

Our results highlight the complex interaction between personal, regional, and global determinants of behavioral response to an epidemic.
Therefore, we emphasize the need to further study heterogeneity in compliance and behavior over both time and space. This can be accomplished both by surveying differences in compliance within and between populations~\citep{Atchison2020}, and by incorporating specific behavioral models into epidemiological models~\citep{Arthur2020,Fenichela2011,Walters2013}.



%\pagebreak
% Models and Methods
{\small
\section*{Models and Methods}



%%% Data %%%%
\paragraph*{Data.} 
We use daily confirmed case data $\vec{X}=(X_1, \ldots, X_T)$ from 12~regions during Jan--Apr 2020. These incidence data summarise the number of individuals $X_t$ tested positive for \sars (using RT-qPCR tests) on each day $t$. 
Data for Wuhan, China, from Jan~10 to Feb~8, retrieved from \citet{Pei2020}. Data for 11 European countries, from Feb~20 to Apr~24, retrieved from \citet{Flaxman2020}.
Where there were multiple sequences of days with zero confirmed cases (e.g. France), we cropped the data to begin with the last sequence so that our analysis focuses on the first sustained outbreak rather than isolated imported cases. 
For official NPI dates see \autoref{table:NPI_dates}.



%%% SEIR model %%%%
\paragraph*{SEIR model.} \label{sec:model}
We model \sars infection dynamics by following the number of susceptible $S$, exposed $E$, reported infected $I_r$, unreported infected $I_u$, and recovered $R$ individuals in a population of size $N$.
This model distinguishes between reported and unreported infected individuals: the reported infected are those that have enough symptoms to eventually be tested and thus appear in daily case reports, to which we fit the model.
This model is inspired by \citet{Li2020} and \citet{Pei2020}, who used a similar model with multiple regions and constant transmission and reporting rates to study \covid dynamics in China and in the continental US.

Susceptible ($S$) individuals become exposed due to contact with reported or unreported infected individuals ($I_r$ or $I_u$) at a rate $\beta_t$ or $\mu \beta_t$, respectively.
The parameter $0 < \mu < 1$ represents the decreased transmission rate from unreported infected individuals, who are often subclinical or even asymptomatic~\citep{Ferretti2020,Thompson2020}.
The transmission rate $\beta_t \ge 0$ may change over  time $t$ due to behavioural changes of both susceptible and infected individuals.
Exposed individuals, after an average latent period of $Z$ days, become reported infected with probability $\alpha_t$ or unreported infected with probability $(1-\alpha_t)$.
The reporting rate $0 < \alpha_t < 1$ may also change over time due to changes in human behaviour.
Infected individuals remain infectious for an average period of $D$ days, after which they either recover, or become ill enough to be quarantined.
In either case, they no longer infect other individuals, and therefore effectively become recovered ($R$).
The model is described by the following set of equations,
\begin{equation} \label{eq:model}
\begin{aligned}
\frac{dS}{dt} & = -\beta_t S \Big(\frac{I_r}{N} + \mu \frac{I_u}{N}\Big) \\
\frac{dE}{dt} & = \beta_t S \Big(\frac{I_r}{N} + \mu \frac{I_u}{N}\Big)  - \frac{E}{Z} \\
\frac{dI_r}{dt} & = \alpha_t \frac{E}{Z} - \frac{I_r}{D} \\
\frac{dI_u}{dt} & = \big(1-\alpha_t\big) \frac{E}{Z} - \frac{I_u}{D} \\
\frac{dR}{dt} & = \frac{I_r}{D} + \frac{I_u}{D} \;,
\end{aligned}
\end{equation}
where $N$ is the population size.
The initial numbers of exposed $E(0)$ and unreported infected $I_u(0)$ are free model parameters (i.e. inferred from the data), whereas the initial number of reported infected and recovered is assumed to be zero, $I_r(0)=R(0)=0$, and the number of susceptible is $S(0)=N-E(0)-I_u(0)$.



%%% Likelihood %%%
\paragraph*{Likelihood function.}

For a given vector $\theta$ of model parameters the \emph{expected} cumulative number of reported infected individuals ($I_r$) until day $t$, following~\autoref{eq:model}, is
\begin{equation} \label{eq:Yt}
Y_t(\theta) = \int_{0}^{t}{\alpha_s \frac{E(s)}{Z} \; ds}, \quad Y_0 = 0.
\end{equation}
We assume that reported infected individuals are confirmed and therefore observed in the daily case report of day $t$ with probability $p_t$ (note that an individual can only be observed once, and that $p_t$ may change over time, but $t$ is a specific date rather than the time elapsed since the individual was infected).
We denote by $X_t$ the \emph{observed} number of confirmed cases in day $t$, and by $\tilde{X}_t$ the cumulative number of confirmed cases until end of day $t$,
\begin{equation} \label{eq:Xsumt}
\tilde{X}_t=\sum_{i=1}^{t}X_i.
\end{equation}
Therefore, at day $t$ the number of reported infected yet-to-be confirmed individuals is
$(Y_t(\theta) - \tilde{X}_{t-1})$.
We assume that $X_t$ conditioned on $\tilde{X}_{t-1}$ is Poisson distributed, such that
\begin{equation} \label{eq:Xt} \begin{aligned}
\Big(X_1 \mid \theta \Big) & \sim \mathit{Poi}\big( Y_1(\theta) \cdot p_1 \big), \\
\Big(X_t \mid \tilde{X}_{t-1}, \theta \Big) & \sim 
\mathit{Poi}\Big( \big(Y_t(\theta) - \tilde{X}_{t-1}\big) \cdot p_t \Big), \quad t=2,\ldots,T.
\end{aligned}\end{equation}

Hence, the \emph{likelihood function} $\mathcal{L}(\theta \mid \vec{X})$ for a parameter vector $\theta$ given the confirmed case data $\vec{X} = (X_1, \ldots, X_T)$ is defined by the probability to observe  $\vec{X}$ given $\theta$,
\begin{equation} \label{eq:likelihood}
\mathcal{L}(\theta \mid \vec{X}) = 
P(\vec{X} \mid \theta) = 
P(X_1 \mid \theta) \cdot P(X_2 \mid \tilde{X_1}, \theta) \cdots P(X_T \mid \tilde{X}_{T-1}, \theta).
\end{equation}



%%% NPI %%%
\paragraph*{NPI model.}
To model non-pharmaceutical interventions (NPIs), we set the start of the NPIs to day $\tau$ and define
\begin{equation} \label{eq:NPI_model}
\beta_t = \begin{cases} 
  \beta, & t < \tau \\ 
  \beta \lambda, & t \ge \tau
\end{cases},
\quad
\alpha_t = \begin{cases} 
  \alpha_1, & t < \tau \\ 
  \alpha_2, & t \ge \tau
\end{cases},
\quad
p_t = \begin{cases} 
  1/9, & t < \tau \\ 
  1/6, & t \ge \tau
\end{cases},
\end{equation}
where $0 < \lambda < 1$.
The values for $p_t$ follow \citet{Li2020}, who estimated the average time between infection and reporting in Wuhan, China, at 9 days before the start of NPIs and 6 days after start of NPIs.

Following \citet{Li2020}, the effective reproduction numbers before and after the start of NPIs are
\begin{equation} \label{eq:Re}
\begin{aligned}
R_1 &= \alpha_1 \beta D + (1-\alpha_1) \mu \beta D, \\
R_2 &= \alpha_2 \lambda \beta D + (1-\alpha_2) \mu \lambda \beta D.
\end{aligned}
\end{equation}
The relative reduction in the effective reproduction number due to NPIs is $\frac{R_1 - R_2}{R_1}$.



%%% Model fitting %%%
\paragraph*{Parameter estimation.}
To estimate the model parameters from the daily case data $\vec{X}$, we apply a Bayesian inference approach.
Model fitting was calibrated for case data up to Mar~28, and then applied to data up to Apr~11 (for Wuhan, China, model fitting was applied for data up to Feb~8.)
We start our model $\Delta t$ days~\citep{Gatto2020} before the outbreak (defined as consecutive days with increasing confirmed cases) in each country.
The model in Eqs.~\ref{eq:model} and~\ref{eq:NPI_model} is parameterised by the vector $\theta$, where
\begin{equation} \label{eq:theta}
\theta=\Big(Z, D, \mu, \beta, \lambda, \alpha_1, \alpha_2, E(0), I_u(0), \Delta t, \tau \Big) \;.
\end{equation} 

The likelihood function is defined in \autoref{eq:likelihood}.
We defined the following prior distributions on the model parameters $P(\theta)$: 
\begin{equation} \label{eq:priors}
\begin{aligned} 
Z & \sim \mathit{Uniform}(2, 5) \\
D & \sim \mathit{Uniform}(2, 5) \\
\mu & \sim \mathit{Uniform}(0.2, 1) \\
\beta & \sim \mathit{Uniform}(0.8, 1.5) \\
\lambda & \sim \mathit{Uniform}(0, 1) \\
\alpha_1, \alpha_2 & \sim \mathit{Uniform}(0.02, 1)\\
E(0) & \sim \mathit{Uniform}(0, 3000) \\
I_u(0) & \sim \mathit{Uniform}(0, 3000) \\
\Delta t & \sim \mathit{Uniform}(1, 5) \\
\tau &\sim \mathit{TruncatedNormal}\Big(\frac{\tau^*+\tau^0}{2}, \frac{\tau^*-\tau^0}{2}, 5, T-2\Big),
\end{aligned}
\end{equation}
where the prior for $\tau$ is a truncated normal distribution shaped so that the date of the first and last NPI, $\tau^0$ and $\tau^*$ (\autoref{table:NPI_dates}), are at minus and plus one standard deviation, and taking values only between 5 and  $T-2$, where $T$ is the number of days in the data $\vec{X}$.
We also tested an uninformative uniform prior, $\mathit{Uniform}(1,T-2)$.
WAIC (\autoref{eq:WAIC}) of a model with this uniform prior was either higher, or lower by less than 2, compared to WAIC of a model with the truncated normal prior.
The uninformative prior resulted in non-negligible posterior probability for unreasonable $\tau$ values, such as Mar~1 in the United~Kingdom. 
We therefore decided to use the more informative truncated normal prior for $\tau$.
Other priors follow \citet{Li2020}, with the following exceptions.
$\lambda$ is used to ensure transmission rates are lower after the start of the NPIs ($\lambda < 1$).
We checked values of $\Delta t$ larger than five days and found they generally produce lower likelihood and unreasonable parameter estimates, and therefore chose $\mathit{Uniform}(1,5)$ as the prior for $\Delta t$.
We also tried to estimate the value of $p_t$ before and after $\tau$, instead of keeping it fixed at $1/9$ and $1/6$. The model with fixed values was supported by the evidence (lower WAIC, see below) in 9 of 12 countries. Moreover, the estimates for Wuhan, China were $1/9$ and $1/6$, as in \citet{Li2020}. 

The posterior distribution of the model parameters $P(\theta \mid \vec{X})$ is estimated using the affine-invariant ensemble sampler for Markov chain Monte Carlo (MCMC)~\citep{Goodman2010}, implemented in the \texttt{emcee} Python package~\citep{Foreman-Mackey2013}.
We use the default configuration using the stretch move with stretch scale parameter $a=2$.
For the main analysis we use 50 chains (or walkers) per region, with 7M samples per chain (no thinning was applied; 6.8M for Wuhan).
The \emph{integrated autocorrelation time} (IAT)~\citep{Foreman-Mackey2013,Goodman2010} was averaged across parameters and chains for each region. 
The average IAT was between 31.9K for Wuhan and 187K for Germany~(\autoref{fig:autocorr}).
We examined the trace plots for $\tau$ in all regions (\autoref{fig:trace}).
All chains seem to converge to the stationary distribution, in most cases before 2M samples. Thus, we discarded the first 2M samples as burn-in samples.
The only exception is Spain, in which a single chain converged at around 6M samples.
We considered this chain as part of the burn-in and removed it from the analysis.
Therefore, 50 chains with 5M samples and IAT between 32K and 187K give an effective sample size between 1,336 and 7,523.
We ran additional chains with 2M samples and a different initialization (i.e. seed). The estimated posterior distributions of $\tau$ were very similar to our main analysis, further increasing our confidence in the convergence of our inference.
For the models with fixed~$\tau$ and without~$\tau$, in some countries, we use less than 7M samples per chain because the IAT converged sooner.
In these cases the effective sample size was at between 2,454 (UK) and 10,954 (Spain) in the fixed~$\tau$ model and between 6,230 (Norway) and 94,339 (Italy) in the model without~$\tau$.



%%% Model comparison %%%
\paragraph*{Model comparison.}
We perform model selection using two methods.
First, we compute WAIC (widely applicable information criterion)~\citep{gelman2013bayesian},
\begin{equation} \label{eq:WAIC}
\begin{aligned}
\mathit{WAIC}(\theta, \vec{X}) &= -2\log\mathbb{E}[\mathcal{L}(\theta \mid \vec{X})] + 2\mathbb{V}[\log\mathcal{L}(\theta \mid \vec{X})]
\end{aligned}
\end{equation}
where $\mathbb{E}[\cdot]$ and $\mathbb{V}[\cdot]$ are the expectation and variance operators taken over the posterior distribution $P(\theta \mid \vec{X})$.
We compare models by reporting their relative WAIC; lower is better~(\autoref{table:WAIC}).

We also plot posterior predictions: we sample 1,000 parameter vectors from the posterior distribution $P(\theta \mid \vec{X})$ fitted to data up to Apr~11, use these parameter vectors to simulate the SEIR model (\autoref{eq:model}) up to Apr~24, and plot the predicted dynamics~(\autoref{fig:ppc}).
Both the accuracy (i.e. overlap of data and prediction) and the precision (i.e. the tightness of the predictions) are good ways to visually compare models.
We also compute the expected posterior RMSE (root mean squared error) of these predictions~(\autoref{table:RMSE}).
} % small

% Declarations
{\small
\section*{Declarations}
\textbf{Ethics approval and consent to participate.} Not applicable.

\textbf{Consent for publication.} Not applicable.

\textbf{Availability of data and materials.} We use Python 3 with NumPy, Matplotlib, SciPy, Pandas, Seaborn, and emcee.
Source code will be publicly available under a permissive open-source license at \href{http://github.com/yoavram-lab/EffectiveNPI}{github.com/yoavram-lab/EffectiveNPI}.
We used freely available data, see source code repository and above \emph{Data} section for details.

\textbf{Competing interests.} The authors declare that they have no competing interests.

\textbf{Funding.} This work was supported in part by the Israel Science Foundation (3811/19 and 552/19, YR). The funding body had no role in designing, performing, or analyzing the study.

\textbf{Authors' contributions.} UO and YR designed the research. IK and YR performed the research and wrote the manuscript. All authors read and approved the final manuscript.

\textbf{Acknowledgements.} 
We thank Yinon M. Bar-On, Lilach Hadany, Oren Kolodny, and Zohar Yakhini. % Tim CD Lucas
} % small

%% Table - estimated parameters
% generated with /python Table_estimated_params.py FOLDER NAME
\begin{landscape}
\begin{table}[]
\centering
\footnotesize{
\input{../figures/Table-estimated-params.tex}
}
\caption{
\textbf{Parameter estimates for different regions.}
See \autoref{eq:model} for model parameters.
All estimates are posterior medians.
75\% and 95\% credible intervals (HDI) are given for $\tau$ in days relative to $\hat{\tau}$.
$\tau^*$ is the official last NPI date (\autoref{table:NPI_dates}).
}
\label{table:estimated-params}
\end{table}
\end{landscape}



%\pagebreak

\section*{References}
\nolinenumbers
\bibliographystyle{agsm}
%\bibliography{/Users/yoavram/Documents/library}
\bibliography{ms}



\pagebreak
\section*{Supplementary Material}
\beginsupplement % https://support.authorea.com/en-us/article/how-to-create-an-appendix-section-or-supplementary-information-1g25i5a/



%%%% Fig - NPI dates %%%
% generated with /python Fig_NPI_dates.py/
\begin{figure}[h]
    \centering
	\includegraphics[width=0.5\textwidth]{../figures/Fig-NPI_dates.pdf}
    \caption{
    \textbf{Official start of non-pharmaceutical interventions.}
	See \autoref{table:NPI_dates} for more details. Wuhan, China is not shown.
    } 
    \label{fig:NPI_dates}
\end{figure}



% Table - RMSE 
% generated with /python Table-RMSE.py FOLDER NAME
\begin{table}[h]
\centering
\caption*{\textbf{RMSE}}
\pgfplotstabletypeset[
    col sep=comma,
    string type,
    every head row/.style={before row=\hline,after row=\hline},
    every last row/.style={after row=\hline},
    ]{../figures/Table-RMSE.csv}
\caption{
\textbf{Posterior RMSE of out-of-sample predictions with the different models.}
	Expected posterior predictive RMSE (root mean squared error) for models  with: no $\tau$ at all, \emph{No}; $\tau$ fixed at the official last NPI date $\tau^*$, \emph{Fixed}; and free parameter $\tau$, \emph{Free}.
	In all cases, the model with free parameter $\tau$ has the lowest RMSE.
	Models were fitted to case data up to Apr~11, 2020, and then used to generate 1,000 predictions up to Apr~24 by sampling model parameters from the posterior distribution.
	These predictions were then compared to the real data using RMSE, and the mean RMSE value is shown in the table for each country and model. 
	Bold values emphasize cases in which the \emph{Free} model has the lowest RMSE.
}
\label{table:RMSE}
\end{table}



% Table - WAIC 
% generated with /python Table-WAIC.py FOLDER NAME
\begin{table}[h]
\centering
\caption*{\textbf{WAIC}}
\pgfplotstabletypeset[
    col sep=comma,
    string type,
    every head row/.style={before row=\hline,after row=\hline},
    every last row/.style={after row=\hline},
    ]{../figures/Table-WAIC.csv} 
\caption{
\textbf{WAIC values for the different models.}
	WAIC (widely applicable information criterion; \autoref{eq:WAIC})~\citep{gelman2013bayesian} values for models  with: no $\tau$ at all, \emph{No}; $\tau$ fixed at the official last NPI date $\tau^*$, \emph{Fixed}; and free parameter $\tau$, \emph{Free}. WAIC values are scaled as a deviance measure: lower values imply higher predictive accuracy and a difference of 2 is a popular threshold for model comparison~\citep{Kass1995}. Bold values emphasize cases in which the \emph{Free} model has the lowest WAIC.
} 
\label{table:WAIC}
\end{table}


%%% Fig - tau summary Mar 28 %%%
% generated with
% > python Fig_tau_summary.py 2020-06-23-Mar28
% > mv ../figures/Fig-tau-summary.pdf ../figures/Fig-summary-Mar28.pdf
\begin{figure}[b!]
    \centering
	\includegraphics[width=0.6\textwidth]{../figures/Fig-summary-Mar28.pdf}
    \caption{
    \textbf{Official vs. effective start of non-pharmaceutical interventions estimated up to Mar~28.}
    	The difference between $\tau$ the effective and $\tau^*$ the official start of NPIs estimated from case data up to Mar~28, 2020, shown for different regions. Here, $\hat{\tau}$ is the marginal posterior median. $\tau^*$ is the last NPI date (a lockdown everywhere by Sweden, see \autoref{table:NPI_dates}). Thin and bold lines show 95\% and 75\% credible intervals (HDI~\citep{Kruschke2014}), respectively. Inference performed similarly to main inference, but with data up to Mar~28 and only 1M samples per chain with 600K burn-in.
    }
    \label{fig:tau-summary-mar28}
\end{figure}
 


%%% Fig PPC
% generated with ppc.sh
\begin{figure*}[h]	
    \centering
	\caption*{Austria}
    \begin{subfigure}{0.45\textwidth}
        \includegraphics[width=\textwidth]{../figures/ppc/fixed/Austria_ppc_long.pdf}
    \end{subfigure}
    ~
    \begin{subfigure}{0.45\textwidth}
        \includegraphics[width=\textwidth]{../figures/ppc/free/Austria_ppc_long.pdf}
    \end{subfigure}
    %%%
	\caption*{Belgium}    
    \begin{subfigure}{0.45\textwidth}
        \includegraphics[width=\textwidth]{../figures/ppc/fixed/Belgium_ppc_long.pdf}
    \end{subfigure}
    ~
    \begin{subfigure}{0.45\textwidth}
        \includegraphics[width=\textwidth]{../figures/ppc/free/Belgium_ppc_long.pdf}
    \end{subfigure}
    %%%
	\caption*{Denmark}    
     \begin{subfigure}{0.45\textwidth}
        \includegraphics[width=\textwidth]{../figures/ppc/fixed/Denmark_ppc_long.pdf}
    \end{subfigure}
    ~
    \begin{subfigure}{0.45\textwidth}
        \includegraphics[width=\textwidth]{../figures/ppc/free/Denmark_ppc_long.pdf}
    \end{subfigure}
	%%%
	\caption*{France}    
    \begin{subfigure}{0.45\textwidth}
        \includegraphics[width=\textwidth]{../figures/ppc/fixed/France_ppc_long.pdf}
    \end{subfigure}
    ~
    \begin{subfigure}{0.45\textwidth}
        \includegraphics[width=\textwidth]{../figures/ppc/free/France_ppc_long.pdf}
    \end{subfigure}
\end{figure*}
    %%%
\begin{figure*}
    \ContinuedFloat 
	\caption*{Germany}    
    \begin{subfigure}{0.45\textwidth}
        \includegraphics[width=\textwidth]{../figures/ppc/fixed/Germany_ppc_long.pdf}
    \end{subfigure}
    ~
    \begin{subfigure}{0.45\textwidth}
        \includegraphics[width=\textwidth]{../figures/ppc/free/Germany_ppc_long.pdf}
    \end{subfigure}
    %%% 
	\caption*{Italy}    
    \begin{subfigure}{0.45\textwidth}
        \includegraphics[width=\textwidth]{../figures/ppc/fixed/Italy_ppc_long.pdf}
    \end{subfigure}
    ~
    \begin{subfigure}{0.45\textwidth}
        \includegraphics[width=\textwidth]{../figures/ppc/free/Italy_ppc_long.pdf}
    \end{subfigure}
    %%% 
	\caption*{Norway}    
    \begin{subfigure}{0.45\textwidth}
        \includegraphics[width=\textwidth]{../figures/ppc/fixed/Norway_ppc_long.pdf}
    \end{subfigure}
    ~
    \begin{subfigure}{0.45\textwidth}
        \includegraphics[width=\textwidth]{../figures/ppc/free/Norway_ppc_long.pdf}
    \end{subfigure}
    %%%  
	\caption*{Spain}     
    \begin{subfigure}{0.45\textwidth}
        \includegraphics[width=\textwidth]{../figures/ppc/fixed/Spain_ppc_long.pdf}
    \end{subfigure}
    ~
    \begin{subfigure}{0.45\textwidth}
        \includegraphics[width=\textwidth]{../figures/ppc/free/Spain_ppc_long.pdf}
    \end{subfigure}
\end{figure*}
    %%%
\begin{figure*}
    \ContinuedFloat 
	\caption*{Sweden}    
    \begin{subfigure}{0.45\textwidth}
        \includegraphics[width=\textwidth]{../figures/ppc/fixed/Sweden_ppc_long.pdf}
    \end{subfigure}
    ~
    \begin{subfigure}{0.45\textwidth}
        \includegraphics[width=\textwidth]{../figures/ppc/free/Sweden_ppc_long.pdf}
    \end{subfigure}
    %%%  
	\caption*{Switzerland}    
    \begin{subfigure}{0.45\textwidth}
        \includegraphics[width=\textwidth]{../figures/ppc/fixed/Switzerland_ppc_long.pdf}
    \end{subfigure}
    ~
    \begin{subfigure}{0.45\textwidth}
        \includegraphics[width=\textwidth]{../figures/ppc/free/Switzerland_ppc_long.pdf}
    \end{subfigure}
    %%%  
	\caption*{United Kingdom}    
    \begin{subfigure}{0.45\textwidth}
        \includegraphics[width=\textwidth]{../figures/ppc/fixed/United_Kingdom_ppc_long.pdf}
    \end{subfigure}
    ~
    \begin{subfigure}{0.45\textwidth}
        \includegraphics[width=\textwidth]{../figures/ppc/free/United_Kingdom_ppc_long.pdf}
    \end{subfigure}
    \caption{    
    \textbf{Posterior prediction plots for 11 European countries.}
    The vertical dashed line represents Apr~11, 2020.
    Circles and stars represent daily case data up to and after Apr~11, respectively.
    Black and white arrows denote the official ($\tau^*$) and effective ($\hat{\tau}$) start of NPIs, respectively.
    Black lines represent a smoothing of the data points using a Savitzky-Golay filter with window length 3. 
    Coloured lines represent posterior predictions from a model with fixed~$\tau$ (blue) and free~$\tau$ (green). Models were fitted with data up to Apr~11. The predictions are generated by drawing 1,000 parameters sets from the posterior distribution, and then generating a daily case count using the SEIR model (\autoref{eq:model}) up to Apr~24. Note the differences in the y-axis scale.
    Posterior predictions with the free~$\tau$ model predict the out-of-sample data well for all countries except Denmark and Sweden, but poorly with the fixed~$\tau$ model.
    The predictions of the model without~$\tau$ (not shown) are even worse.
    }
    \label{fig:ppc}	    
\end{figure*}



%%% Fig Joint density plots
\begin{figure}[p]	
    \centering
    \begin{subfigure}{0.32\textwidth}
        \includegraphics[width=\textwidth]{../figures/joint/Austria_joint.pdf}
    \end{subfigure}
    ~%%%
    \begin{subfigure}{0.32\textwidth}
        \includegraphics[width=\textwidth]{../figures/joint/Belgium_joint.pdf}
    \end{subfigure}
    ~%%%
    \begin{subfigure}{0.32\textwidth}
        \includegraphics[width=\textwidth]{../figures/joint/Denmark_joint.pdf}
    \end{subfigure}
	~%%%
    \begin{subfigure}{0.32\textwidth}
        \includegraphics[width=\textwidth]{../figures/joint/France_joint.pdf}
    \end{subfigure}
    ~%%%
    \begin{subfigure}{0.32\textwidth}
        \includegraphics[width=\textwidth]{../figures/joint/Germany_joint.pdf}
    \end{subfigure}
    ~%%% 
    \begin{subfigure}{0.32\textwidth}
        \includegraphics[width=\textwidth]{../figures/joint/Italy_joint.pdf}
    \end{subfigure}
    ~%%% 
	\begin{subfigure}{0.32\textwidth}
        \includegraphics[width=\textwidth]{../figures/joint/Norway_joint.pdf}
    \end{subfigure}
    ~%%%  
	\begin{subfigure}{0.32\textwidth}
        \includegraphics[width=\textwidth]{../figures/joint/Spain_joint.pdf}
    \end{subfigure}
    ~%%%
	\begin{subfigure}{0.32\textwidth}
        \includegraphics[width=\textwidth]{../figures/joint/Sweden_joint.pdf}
    \end{subfigure}
    ~%%%  
	\begin{subfigure}{0.32\textwidth}
        \includegraphics[width=\textwidth]{../figures/joint/Switzerland_joint.pdf}
    \end{subfigure}
    ~%%%  
	\begin{subfigure}{0.32\textwidth}
        \includegraphics[width=\textwidth]{../figures/joint/United_Kingdom_joint.pdf}
    \end{subfigure}
    ~%%%  
	\begin{subfigure}{0.32\textwidth}
        \includegraphics[width=\textwidth]{../figures/joint/Wuhan_joint.pdf}
    \end{subfigure}
    \caption{
    \textbf{Joint posterior density plots for $\tau$ and $\lambda$.} The high values of $\lambda$ estimated in Denmark and Sweden reduce the effect of the NPIs thereby making the inference of $\tau$ more difficult, resulting in wide posterior distributions. A correlation between the parameters is also evident in Norway. In Belgium and in Wuhan a later $\tau$ gives a wide estimate for $\lambda$. In comparison, Austria, France, Germany, Italy, Spain, Switzerland, and UK have a narrow joint distribution.
    }
    \label{fig:joint}
\end{figure}



\begin{figure}[h]
    \centering
	\includegraphics[width=\textwidth]{../figures/Fig_Re2.pdf}
    \caption{
    \textbf{Under-estimation of impact of NPIs increases with the delay in their effective start.} The x-axis shows the days between the estimated effective start of NPIs, $\hat\tau$, and the official date, $\tau^*$. The y-axis shows the error in estimation of the impact of NPIs when assuming they start at their official date, i.e. the difference between the median estimate from a model with fixed~$\tau$ minus the estimate from a model with free~$\tau$, the y-axis and x-axis of \autoref{fig:Re}, respectively.
    Impact of NPIs defined as $\frac{R_1 - R_2}{R_1}$, where $R_1$ and $R_2$ are the effective reproduction numbers before and after the NPI, respectively (\autoref{eq:Re}).
	The dashed black line shows a linear regression, slope=-0.0313 with 95\% CI [-0.051, -0.012], $R^2=0.554$.
    } 
    \label{fig:Re2}
\end{figure}



%%% Fig Fr-vs-Es
% generated with Fig-fr-vs-es.ipynb
\begin{figure}[h]
    \centering
	\includegraphics[width=\textwidth]{../figures/Fig-fr-vs-es.pdf}
    \caption{
    \textbf{\covid daily confirmed cases in France and Spain.}
    Number of cases proportional to population size (as of 2018). 
    Vertical line shows Mar~8, the effective start of NPIs $\hat{\tau}$ in both countries.
    Data from~\citet{Flaxman2020}.
    } 
    \label{fig:fig-fr-vs-es}
\end{figure}



%%% Fig Autocorrelation
% generated with Fig_autocorr.py
\begin{figure}[h]
    \centering
	\includegraphics[width=\textwidth]{../figures/Fig-autocorr.pdf}
    \caption{
    \textbf{Integrated autocorrelation time (IAT)}~\citep{Foreman-Mackey2013,Goodman2010}, averaged across model parameters and MCMC chains, as a function of the number of samples in the chains.
    IAT is less than 187K in all cases, while chain length is 5M after burn-in period of 2M. With 50 chains per region, this gives a at least 1,335 uncorrelated samples for estimating the posterior distribution.
    } 
    \label{fig:autocorr}
\end{figure}



%%% Fig Trace plots
\begin{figure}[p]	
    \centering
    \begin{subfigure}{0.32\textwidth}
        \includegraphics[width=\textwidth]{../figures/traces/Austria_trace.pdf}
    \end{subfigure}
    ~%%%
    \begin{subfigure}{0.32\textwidth}
        \includegraphics[width=\textwidth]{../figures/traces/Belgium_trace.pdf}
    \end{subfigure}
    ~%%%
    \begin{subfigure}{0.32\textwidth}
        \includegraphics[width=\textwidth]{../figures/traces/Denmark_trace.pdf}
    \end{subfigure}
	~%%%
    \begin{subfigure}{0.32\textwidth}
        \includegraphics[width=\textwidth]{../figures/traces/France_trace.pdf}
    \end{subfigure}
    ~%%%
    \begin{subfigure}{0.32\textwidth}
        \includegraphics[width=\textwidth]{../figures/traces/Germany_trace.pdf}
    \end{subfigure}
    ~%%% 
    \begin{subfigure}{0.32\textwidth}
        \includegraphics[width=\textwidth]{../figures/traces/Italy_trace.pdf}
    \end{subfigure}
    ~%%% 
	\begin{subfigure}{0.32\textwidth}
        \includegraphics[width=\textwidth]{../figures/traces/Norway_trace.pdf}
    \end{subfigure}
    ~%%%  
	\begin{subfigure}{0.32\textwidth}
        \includegraphics[width=\textwidth]{../figures/traces/Spain_trace.pdf}
    \end{subfigure}
    ~%%%
	\begin{subfigure}{0.32\textwidth}
        \includegraphics[width=\textwidth]{../figures/traces/Sweden_trace.pdf}
    \end{subfigure}
    ~%%%  
	\begin{subfigure}{0.32\textwidth}
        \includegraphics[width=\textwidth]{../figures/traces/Switzerland_trace.pdf}
    \end{subfigure}
    ~%%%  
	\begin{subfigure}{0.32\textwidth}
        \includegraphics[width=\textwidth]{../figures/traces/United_Kingdom_trace.pdf}
    \end{subfigure}
    ~%%%  
	\begin{subfigure}{0.32\textwidth}
        \includegraphics[width=\textwidth]{../figures/traces/Wuhan_trace.pdf}
    \end{subfigure}
    \caption{
    \textbf{Trace plots for $\tau$.}
    The value of $\tau$ at for sequential samples of 50 chains per region. To facilitate visualization, chains were thinned 1:10,000 for the trace plots, but not for posterior estimation. Line transparency was set at $\alpha=0.1$ and chains cycle through different colors. In France and Austria some chains converged relatively late but before the burn-in period (2M iterations). In Spain, a single chain converged very late (after roughly 6M iterations) and was removed from further analysis. The y-axis range is the support of the prior on $\tau$.  
    }
    \label{fig:trace}	    
\end{figure}



\end{document}  