\documentclass[12pt]{extarticle}
\usepackage{geometry}
\geometry{
a4paper,
total={170mm,257mm},
left=20mm,
top=20mm,
headheight=12pt
}

\usepackage[parfill]{parskip} % Activate to begin paragraphs with an empty line rather than an indent
\usepackage{graphicx} % Use pdf, png, jpg, or eps§ with pdflatex; use eps in DVI mode
% TeX will automatically convert eps --> pdf in pdflatex		

\usepackage{amssymb,amsmath,amsthm}
\usepackage{commath}
\usepackage{longtable}
\usepackage[hyphens]{url}
\usepackage[dvipsnames]{xcolor}
\usepackage[unicode=true,colorlinks=true,urlcolor=Blue,citecolor=black,linkcolor=black]{hyperref}
\def\equationautorefname~#1\null{Eq.~(#1)\null}
\PassOptionsToPackage{hyphens}{url} % url is loaded by hyperref
\usepackage{authblk}
\usepackage{lipsum}
\usepackage{multicol}
\usepackage{titlesec}	
\usepackage[font=small,labelfont=bf]{caption}
\usepackage{enumitem}
\usepackage{soul}
\usepackage{booktabs}
\usepackage{pgfplotstable} % http://mirrors.ctan.org/graphics/pgf/contrib/pgfplots/doc/pgfplotstable.pdf
\usepackage{subcaption}

%SetFonts
% newtxtext+newtxmath
\usepackage{newtxtext} %loads helv for ss, txtt for tt
\usepackage{amsmath}
\usepackage[bigdelims]{newtxmath}
\usepackage[T1]{fontenc}
\usepackage{textcomp}
%SetFonts

% less space before sections 
% \@startsection {NAME}{LEVEL}{INDENT}{BEFORESKIP}{AFTERSKIP}{STYLE} 
%            optional * [ALTHEADING]{HEADING} 
\makeatletter
 \renewcommand\section{\@startsection {section}{1}{\z@}%
     {-2.5ex \@plus -1ex \@minus -.2ex}%
     {1.3ex \@plus.2ex}%
    {\Large\bfseries}}
    
% Species names
%% Meta-Command for defining new species macros
\usepackage{xspace}

\newcommand{\species}[3]{%
  \newcommand{#1}{\gdef#1{\textit{#3}\xspace}\textit{#2}\xspace}}
  
\species{\yeast}{Saccharomyces cerevisiae}{S.~cerevisiae}
\species{\calbicans}{Candida albicans}{C.~albicans}
\species{\cneoformans}{Cryptococcus neoformans}{C.~neoformans}

% line numbers
 \usepackage[displaymath, mathlines]{lineno}
 \renewcommand\linenumberfont{\normalfont\small\sffamily}
\linenumbers
% \modulolinenumbers[2]

% Yoav & Lee commands
\newcommand*{\tr}{^\intercal}
\let\vec\mathbf
\newcommand{\matrx}[1]{{\left[ \stackrel{}{#1}\right]}}
\newcommand{\diag}[1]{\mbox{diag}\matrx{#1}}
\newcommand{\goesto}{\rightarrow}
\newcommand{\dspfrac}[2]{\frac{\displaystyle #1}{\displaystyle #2} }
\newtheorem{theorem}{Theorem}
\newtheorem{corollary}{Corollary}
\newtheorem{lemma}{Lemma}
\newtheorem{remark}{Remark}
\newtheorem{result}{Result}
\renewcommand\qedsymbol{} % no square at end of proof
\newcommand{\cl}{\mathbf{L}}
\newcommand{\cj}{\mathbf{J}}
\newcommand{\ci}{I}

% Supplementary
\newcommand{\beginsupplement}{%
      	\setcounter{table}{0}
        \renewcommand{\thetable}{S\arabic{table}}%
        \setcounter{figure}{0}
        \renewcommand{\thefigure}{S\arabic{figure}}%
}
     
% NatBib
\usepackage[numbers,super,sort&compress]{natbib}
%\usepackage[round,colon,authoryear]{natbib}
\renewcommand{\bibsection}{}
\renewcommand{\bibfont}{\small}

% unbreakable dashes
\usepackage[shortcuts]{extdash}

% Title page
\title{Late and early effective start of non-pharmaceutical interventions inferred during COVID-19 outbreaks }

% Authors
\renewcommand\Affilfont{\small}

\author[a]{Ilia Kohanovski}
\author[b,c]{Uri Obolski}
\author[a,*]{Yoav Ram}

\affil[a]{School of Computer Science, Interdisciplinary Center Herzliya, Herzliya 4610101, Israel}
\affil[b]{School of Public Health, Tel Aviv University, Tel Aviv 6997801, Israel}{
\affil[c]{Porter School of the Environment and Earth Sciences, Tel Aviv University, Tel Aviv 6997801, Israel}
\affil[*]{Corresponding author: yoav@yoavram.com}

% Document
\begin{document}
\maketitle

% Abstract
\begin{abstract}
During February and March 2020, several countries implemented a variety of non-pharmaceutical interventions with variable schedules to control the COVID\=/-19 pandemic caused by the SARA\=/CoV\=/2 virus.
Overall, these interventions have successfully reduced the ...
\end{abstract}

\pagebreak
% Introduction
\section*{Introduction}

The COVID-19 pandemic has resulted in implementation of extreme non-pharmaceutical interventions (NPIs) in many affected countries. These interventions, from social distancing to lockdowns, are applied in a rapid and widespread fashion.
The NPIs are designed and assessed using epidemiological models, which follow the dynamics of the viral infection to forecast the effect of different mitigation and suppression strategies on the levels of infection, hospitalization, and fatality.
These epidemiological models usually assume that the effect of NPIs on disease transmission begins at the officially declared date (e.g.~\citet{Li2020,Gatto2020,Flaxman2020}).

Adoption of public health recommendations is often critical for effective response to infectious diseases, and has been studied in the context of HIV~\citep{Kaufman2014} and vaccination~\citep{Dunn2015,Wiyeh2018}, for example.
However, behavioral and social change does not occur immediately, but rather requires time to diffuse in the population through media, social networks, and social interactions. 
Moreover, compliance to NPIs may differ between different interventions and between people.
For example, in a survey of 2,108 adults in the UK during Mar 2020, \citet{Atchison2020} found that those over 70 years old were more likely to adopt social distancing than young adults (18-34 years old), and that those with lower income were less likely to be able to work from home and to self-isolate.
Similarly, compliance to NPIs may be impacted by personal experiences. \citet{Smith2020} have surveyed 6,149 UK adults in late April and found that people who believe they have already had COVID\=/19 are more likely to think they are immune, and less likely to comply with social distancing measures. 
Compliance may also depend on risk perception as perceived by the the number of domestic cases or even by reported cases in other regions and countries.
Interestingly, the perceived risk of COVID\=/19 infection has likely caused a reduction in the number of influenza-like illness cases in the US starting from mid-February~\citep{Zipfel2020}.

Here, we hypothesize that there is a significant difference between the official start of NPIs and their adoption by the public and therefore their effect on transmission dynamics.
We use a \textit{Susceptible-Exposed-Infected-Recovered} (SEIR) epidemiological model and \textit{Markov Chain Monte Carlo} (MCMC) parameter estimation framework to estimate the effective start date of NPIs from publicly available COVID-19 case data in several geographical regions.
We compare these estimates to the official dates and find both delayed and advanced effect of NPIs on COVID\=/19 transmission dynamics.
We conclude by demonstrating how differences between the official and effective start of NPIs can confuse assessments of the effectiveness of the NPIs in a simple epidemic control framework. % TODO explain why important



%\pagebreak
% Models and Methods
\section*{Models and Methods}



%%% Data %%%%
\paragraph*{Data.} 
We use daily confirmed case data $\vec{X}=(X_1, \ldots, X_T)$ from several different countries. These incidence data summarize the number of individuals $X_t$ tested positive for SARS\=/CoV\=/2 RNA (using RT-qPCR) at each day $t$.
Data for Wuhan, China retrieved from \citet{Pei2020}, data for 11 European countries retrieved from \citet{Flaxman2020}. 
Regions in which there were multiple sequences of days with zero confirmed cases (e.g. France), we cropped the data to begin with the last sequence so that our analysis focuses on the first sustained outbreak rather than isolated imported cases. 
For dates of official NPI dates see \autoref{table:table1}.


 
% Table 1 %%%
\begin{table}[h]
\centering
\pgfplotstabletypeset[
    col sep=comma,
    string type,
    every head row/.style={before row=\hline,after row=\hline},
    every last row/.style={after row=\hline},
    ]{../data/NPI_dates.csv}
\caption{
\textbf{Official start of non-pharmaceutical interventions.}
The date of the first intervention is for a ban of public events, or encouragement of social distancing, or for school closures.
In all countries except Sweden, the date of the last intervention is for a lockdown. In Sweden, where a lockdown was not ordered during the studied dates, the last date is for school closures. Dates for European countries from \citet{Flaxman2020}, date for Wuhan, China from \citet{Pei2020}.
}
\label{table:table1}
\end{table}



%%% SEIR model %%%%
\paragraph*{SEIR model.}
We model SARS\=/CoV\=/2 infection dynamics by following the number of susceptible $S$, exposed $E$, reported infected $I_r$, and unreported infected $I_u$ individuals in a population of size $N$.
This model distinguishes between reported and unreported infected individuals: the reported infected are those that have enough symptoms to eventually be tested and thus appear in daily case reports, to which we fit the model.

Susceptible ($S$) individuals become exposed due to contact with reported or unreported infected individuals ($I_r$ or $I_u$) at a rate $\beta_t$ or $\mu \beta_t$.
The parameter $0 < \mu < 1$ represents the decreased transmission rate from unreported infected individuals, who are often subclinical or even asymptomatic.
The transmission rate $\beta_t \ge 0$ may change over  time $t$ due to behavioural changes of both susceptible and infected individuals.
Exposed individuals, after an average incubation period of $Z$ days, become reported infected with probability $\alpha_t$ or unreported infected with probability $(1-\alpha_t)$.
The reporting rate $0 < \alpha_t < 1$ may also change over time due to changes in human behavior.
Infected individuals remain infectious for an average period of $D$ days, after which they either recover, or becomes ill enough to be quarantined.
They therefore no longer infect other individuals, and the model does not track their frequency.
The model is described by the following equations:

\begin{equation} \label{eq:model}
\begin{aligned}
\frac{dS}{dt} & = -\beta_t S \frac{I_p}{N} - \mu \beta_t S \frac{I_s}{N} \\
\frac{dE}{dt} & = \beta_t S \frac{I_p}{N} + \mu \beta_t S \frac{I_s}{N}  - \frac{E}{Z} \\
\frac{dI_r}{dt} & = \alpha_t \frac{E}{Z} - \frac{I_r}{D} \\
\frac{dI_u}{dt} & = (1-\alpha_t) \frac{E}{Z} - \frac{I_r}{D} .
\end{aligned}
\end{equation}
The initial numbers of exposed $E(0)$ and unreported infected $I_u(0)$ are considered model parameters, whereas the initial number of reported infected is assumed to be zero $I_r(0)=0$, and the number of susceptible is $S(0)=N-E(0)-I_u(0)$.
This model is inspired by \citet{Li2020} and \citet{Pei2020}, who used a similar model with multiple regions and constant transmission $\beta$ and reporting rate $\alpha$ to infer COVID\=/19 dynamics in China and the continental US, respectively.



%%% Likelihood %%%
\paragraph*{Likelihood function.}
The \emph{expected} cumulative number of reported infected individuals until day $t$ is 
\begin{equation} \label{eq:Yt}
Y_t=\int_{0}^{t}{\alpha_s \frac{E(s)}{Z} \; ds}, \quad Y_0 = 0.
\end{equation}
We assume that reported infected individuals are confirmed and therefore observed in the daily case report of day $t$ with probability $p_t$ (note that an individual can only be observed once, and that $p_t$ may change over time, but $t$ is a specific date rather than the time elapsed since the individual was infected).
Hence, we assume that the number of confirmed cases in day $t$ is binomially distributed,
$$
X_t \sim \mathit{Bin}\big(n_t, p_t\big),
$$
where $n_t$ is the \emph{realized} (rather than expected) number of reported infected individuals yet to appear in daily reports by day~$t$.
The cumulative number of confirmed cases until day $t$ is
$$
\tilde{X}_t=\sum_{i=1}^{t}X_i, \quad X_0=0.
$$
Given $\tilde{X}_{t-1}$, we assume $n_t$ is Poisson distributed,
$$
\big(n_t \mid \tilde{X}_{t-1}\big) \sim \mathit{Poi}\Big( Y_t - \tilde{X}_{t-1} \Big), 
\quad n_1 \sim \mathit{Poi}(Y_1).
$$ 
Therefore, $\big(X_t \mid \tilde{X}_{t-1} \big)$ is a binomial conditioned on a Poisson, which reduces to a Poisson with
\begin{equation} \label{eq:P(Xt|Xt-1)}
\big(X_t \mid \tilde{X}_{t-1} \big) \sim \mathit{Poi}\Big( \big( Y_t - \tilde{X}_{t-1} \big) \cdot p_t \Big), 
\quad X_1 \sim \mathit{Poi}(Y_1 \cdot p_1).
\end{equation}

For given vector $\theta$ of model parameters (\autoref{eq:theta}), we compute the expected cumulative number of reported infected individuals $\{Y_t\}_{t=1}^{T}$ for each day (\autoref{eq:Yt}). Then, since $\tilde{X}_{t-1}$ is a function of $X_1, \ldots, X_{t-1}$, we can use \autoref{eq:P(Xt|Xt-1)} to write the probability to observe the confirmed case data $\vec{X} = (X_1, \ldots, X_T)$ as 
\begin{equation} \label{eq:likelihood}
\mathbb{L}(\theta \mid \vec{X}) = P(\vec{X} \mid \theta) = P(X_1 \mid \theta) P(X_2 \mid \tilde{X_1}, \theta) \cdots P(X_T \mid \tilde{X}_{T-1}, \theta).
\end{equation}
This defines a \emph{likelihood function} $\mathbb{L}(\theta \mid \vec{X})$ for the parameter vector $\theta$ given the data $\vec{X}$.



%%% NPI %%%
\paragraph*{NPI model.}
To model non-pharmaceutical interventions (NPIs), we set the beginning of the NPIs to day $\tau$ and define
\begin{equation} \label{eq:NPI_model}
\beta_t = \begin{cases} 
  \beta, & t < \tau \\ % TODO make sure < and not \le, and the next line \ge and not >
  \beta \lambda, & t \ge \tau
\end{cases},
\quad
\alpha_t = \begin{cases} 
  \alpha_1, & t < \tau \\ % TODO make sure < and not \le, and the next line \ge and not >
  \alpha_2, & t \ge \tau
\end{cases},
\quad
p_t = \begin{cases} 
  1/9, & t < \tau \\ % TODO make sure < and not \le, and the next line \ge and not >
  1/6, & t \ge \tau
\end{cases},
\end{equation}
where $0 < \lambda < 1$.
The values for $p_t$ follow \citet{Li2020}, who estimated the average time between infection and reporting in Wuhan, China, at 9 days before the start of NPIs (Jan 23, 2020) and 6 days after start of NPIs.
The parameter $\tau$ is then added to the parameter vector $\theta$ (\autoref{eq:theta}).



%%% Model fitting %%%
\paragraph*{Parameter estimation.}
To estimate the parameters of our model from the data $\vec{X}$, we apply a Bayesian inference approach.
We start our model $\Delta t$ days before the outbreak (defined as consecutive days with increasing confirmed cases) in each country~\citep{Gatto2020}.
The model in \autoref{eq:model} is parameterized by the vector $\theta$, where
\begin{equation} \label{eq:theta}
\theta=\Big(Z, D, \mu, \{\beta_t\}, \{\alpha_t\}, \{p_t\}, E(0), I_u(0)\Big), \tau, \Delta t.
\end{equation} 

The likelihood function is defined in \autoref{eq:likelihood}.
We define the following prior distributions on the model parameters $P(\theta)$: 
\begin{equation} \label{eq:priors}
\begin{aligned} % TODO verify before submitting
Z & \sim \mathit{Uniform}(2, 5) \\
D & \sim \mathit{Uniform}(2, 5) \\
\mu & \sim \mathit{Uniform}(0.2, 1) \\
\beta & \sim \mathit{Uniform}(0.8, 1.5) \\
\lambda & \sim \mathit{Uniform}(0, 1) \\
\alpha_1, \alpha_2 & \sim \mathit{Uniform}(0.02, 1)\\
E(0) & \sim \mathit{Uniform}(0, 3000) \\
I_u(0) & \sim \mathit{Uniform}(0, 3000) \\
\tau &\sim \mathit{TruncatedNormal}(\tau^*, 5, 1, T-2),
\end{aligned}
\end{equation}
where $\mathit{TruncatedNormal}(\mu, \sigma, a, b)$ is a truncated normal distribution with mean $\mu$ and standard deviation $\sigma$ taking values between $a$ and $b$; $T$ is the number of days in the data $\vec{X}$; and $\tau^*$ is the official start of the NPI.
Most priors follow \citet{Li2020}, with the following exceptions.
$\lambda$ is used to ensure transmission rates are lower after the start of the NPIs ($\lambda < 1$).
We checked values of $\Delta t$ larger than five days and found they generally produce lower likelihood and unreasonable parameter estimates.
For the effective start of NPIs $\tau$ we have also tested an uninformative uniform prior $U(1,T-1)$. DIC (see definition below) was lower for the truncated normal prior in most countries, except \hl{Germany?}. More importantly, the uninformative prior could result in non-negligible posterior probability for unreasonable $\tau$ values, such as Mar~1 in the United~Kingdom (this was due to MCMC chains being stuck in low posterior regions of the parameter space).
We therefore decided to use the more informative truncated normal prior.

The posterior distribution of the model parameters $P(\theta \mid \vec{X})$ is then estimated using an \textit{affine-invariant ensemble sampler for Markov chain Monte Carlo} (MCMC) implemented in the \texttt{emcee} Python package~\citep{Foreman-Mackey2013}.
The maximum a posteriori



%%% Model selection %%%
\paragraph*{Model selection.}
We perform model selection using DIC (deviance information criterion)~\citep{Spiegelhalter2002},
\begin{equation} \label{eq:DIC}
\begin{aligned}
DIC(\theta, \vec{X}) &= 2\mathbb{E}[D(\theta)] - D(\mathbb{E}[\theta])   \\
&= 2 \log\mathcal{L}(\mathbb{E}[\theta] \mid \vec{X}) - 4\mathbb{E}[\log\mathcal{L}(\theta \mid \vec{X})],
\end{aligned}
\end{equation}
where $D(\theta)=-2\log\mathcal{L}(\theta \mid \vec{X})$ is the Bayesian deviance, and expectations $\mathbb{E}[\cdot]$ are taken over the posterior distribution $P(\theta \mid \vec{X})$.
We compare models by reporting their relative DIC; lower is better.



%%% Source code %%%
\paragraph*{Source code.} 
We use Python 3 (Anaconda) with the NumPy, Matplotlib, SciPy, Pandas, Seaborn, and emcee packages.
All source code will be publicly available under a permissive open-source license at \href{http://github.com/yoavram-lab/EffectiveNPI}{github.com/yoavram-lab/EffectiveNPI}.



%%% Fig 1 %%%
\begin{figure}[h]
    \centering
	\includegraphics[width=0.6\textwidth]{../figures/Fig1.pdf}
    \caption{
    \textbf{Official and effective start of non-pharmaceutical interventions.}
    	The difference between $\hat{\tau}$ the effective and $\tau^*$ the official start of NPI is shown for different regions. The effective NPI dates in Italy and Wuhan are significantly delayed compared to the official dates, whereas in Denmark, France, Spain, and Germany, the effective date is earlier than the official date.
	$\hat{\tau}$ is the posterior median, see \hl{TABLE2}. $\tau^*$ is the last NPI date, see \autoref{table:table1}. Thin and bold lines show 95\% and 75\% credible intervals (area in which $P(|\tau - \hat{\tau}| \mid \vec{X}) = 0.95$ and $0.75$.)
    }
    \label{fig:fig1}
\end{figure}



%\pagebreak
% Results
\section*{Results}

Several studies have described the effects of non-pharmaceutical interventions in different geographical regions~\citep{Flaxman2020,Gatto2020,Li2020}. 
These studies have assumed that the parameters of the epidemiological model change at a specific date, as in \autoref{eq:NPI_model}, and set the change date $\tau$ to the official NPI date $\tau^*$ (\autoref{table:table1}).
They then fit the model once for time $t<\tau^*$ and once for time $t \ge \tau^*$.
For example, \citet{Li2020} estimate the dynamics in China before and after $\tau^*$ at Jan 23. Thereby, they effectively estimate $(\beta, \alpha_1)$ and $(\lambda, \alpha_2)$ separately.
Here we estimate the posterior distribution $P(\tau \mid \vec{X})$ of the \emph{effective} start date of the NPIs by jointly estimating $\tau, \beta, \lambda, \alpha_1, \alpha_2$ on the entire data per region (e.g. Italy, Austria), rather than splitting the data at $\tau^*$.
We then estimate the maximum a posteriori (MAP) estimate $\hat{\tau}=\mathit{argmax}_{\tau}{P(\tau \mid \vec{X})}$. \hl{using median or argmax?}

We find that a model that considers an NPI (\autoref{eq:NPI_model}) is a better fit to the data than a model without an NPI, i.e. with constant $\beta$ and $\alpha$ (\hl{$\Delta DIC > ?$ for all regions}.)
We compare the official $\tau^*$ and effective $\hat{\tau}$ start of NPIs and find that in most regions the effective start of NPI  significantly differs from the official date (\autoref{fig:fig1}): the \hl{95\%} credible interval on $\hat{\tau}$ does not include $\tau^*$, and the DIC of the model with free $\tau$ parameter is lower than that of a model with a fixed $\tau \equiv \tau^*$ (\hl{$\Delta DIC > ?$}.) The exception that proves the rule is \hl{Switzerland}, where the effective and official dates are the same. Another important exception is the \hl{United~Kingdom?} % TODO need more data? Maybe not such an exception?

In the following, we describe our findings on delayed and advanced start of NPI in detail.



%%% Delayed
\paragraph*{Delayed effective start of NPI.}
In both Wuhan, China, and in Italy we find that our estimated effective start of NPI $\hat{\tau}$ is significantly later than the official date $\tau^*$ (\autoref{fig:fig1}). 

In Italy, the first case officially confirmed on Feb~21, a lockdown was declared in Northern Italy on Mar~8, with social distancing implemented in the rest of the country, and the lockdown was extended to the entire nation on Mar 11~\citep{Gatto2020}.
That is, the official date $\tau^*$ is either Mar~8 or~11.
However, we estimate the effective date $\hat{\tau}$ at Mar~16 ($\pm$0.7 days 95\% CI ; \autoref{fig:fig2}).
Similarly, in Wuhan, China, a lockdown was ordered on Jan 23~\citep{Li2020}, but we estimate  the effective start of NPIs to be several days layer at around Mar~2 ($\pm$2.65 days 95\% CI \autoref{fig:fig2}).


%%% Fig 2 %%%
\begin{figure}[h]
    \centering
    % Wuhan
    % τ_hat = 19.33
	% +-2.65 days for 0.95 quantile
    \begin{subfigure}{0.45\textwidth}
        \includegraphics[width=\textwidth]{../figures/Fig2a.pdf}
    \end{subfigure}
  	~
	% Italy
	% τ_hat = 24.74
	% +-0.71 days for 0.95 quantile
    \begin{subfigure}{0.45\textwidth}
        \includegraphics[width=\textwidth]{../figures/Fig2b.pdf}
    \end{subfigure}
    \caption{
	\textbf{Delayed effect of non-pharmaceutical interventions in Italy and Wuhan, China.}
    Posterior distribution of $\tau$, the effective start date of NPI, is shown as a histogram of MCMC samples. Red line shows the official last NPI date $\tau^*$. Black line shows the estimated $\hat{\tau}$. Shaded area shows a 95\% credible interval (area in which $P(|\tau - \hat{\tau}| \mid \vec{X}) = 0.95$). 
    }
    \label{fig:fig2}
\end{figure}



%%% Advanced
\paragraph*{Advanced effective start of NPIs.}
In contrast, in some regions we estimate an effective start of NPIs $\hat{\tau}$ that is \emph{earlier} then the official date $\tau^*$ (\autoref{fig:fig1}).
In Spain, social distancing was encouraged starting on Mar~8~\citep{Flaxman2020}, but mass gatherings still occurred on Mar~8, including a march of 120,000 people for the \href{https://www.nytimes.com/2020/03/13/world/europe/spain-coronavirus-emergency.html}{International Women's Day}, and a  football match between \href{https://www.espn.com/soccer/match?gameId=550350}{Real Betis and Real Madrid} (2:1) with a crowd of 50,965 in Seville.
A national lockdown was only announced on Mar~14~\citep{Flaxman2020}.
Nevertheless, we estimate the effective start of NPI $\hat{\tau}$ at Mar~8 or~9 ($\pm$2.96 95\%CI), rather than Mar~14 (\autoref{fig:fig3}).

Similarly, in France the official lockdown started at Mar~17 ($\tau^*$), with initial NPIs at Mar 13~\citep{Flaxman2020}. 
However, we estimate the effective start of NPIs $\hat{\tau}$ at Mar~8 ($\pm$5.9 days 95\% CI). Although the credible interval is wide, spanning from Mar~2 to Mar~13, the official lockdown start at Mar~17 is later still (\autoref{fig:fig3}).

Interestingly, the effective start of NPIs $\hat{\tau}$ in both France and Spain is estimated at Mar~8, although the official dates are differ by three days. Moreover, the number of daily cases was similar until Mar~8 in both countries, but diverged by Mar~13, reaching significantly higher numbers in Spain (\autoref{fig:fig-fr-vs-es}).



%%% Fig 3 %%%
\begin{figure}[h]
    \centering
    % France
    % τ_hat = 12.38
	% +-5.90 days for 0.95 quantile
    \begin{subfigure}{0.45\textwidth}
        \includegraphics[width=\textwidth]{../figures/Fig3a.pdf}
    \end{subfigure}
    ~
    % Spain
    % τ_hat = 13.95
	% +-2.96 days for 0.95 quantile
    \begin{subfigure}{0.45\textwidth}
        \includegraphics[width=\textwidth]{../figures/Fig3b.pdf}
    \end{subfigure}
    \caption{
    \textbf{Advanced effect of non-pharmaceutical interventions in France and Spain.}
    Posterior distribution of $\tau$, the effective start date of NPI, is shown as a histogram of MCMC samples. Red line shows the official last NPI date $\tau^*$. Black line shows the estimated $\hat{\tau}$. Shaded area shows a 95\% credible interval (area in which $P(|\tau - \hat{\tau}| \mid \vec{X}) = 0.95$). 
	}
	\label{fig:fig3}
\end{figure}


%%% Exact
\paragraph*{The exception that proves the rule.}
We find one case in which the official and effective dates match: Switzerland ordered a national lockdown on Mar~20, after banning public evens and closing schools on Mar~13 and~14~\citep{Flaxman2020}.
Indeed, our MAP estimate $\hat{\tau}$ is Mar~20, and the posterior distribution shows two density peaks: a smaller one between Mar~10 and Mar~14, and a taller one between Mar~17 and Mar~22. It's also worth mentioning that Switzerland was the first to mandate self isolation of confirmed cases~\citep{Flaxman2020}.

%%% Assessment
\paragraph*{Effect of delays and advances of real-time assessment.}

The success of non-pharmaceutical interventions is assessed by health officials using various metrics, such as the decline in the growth rate of daily cases. These assessments are made a specific number of days after the intervention began, to accommodate for the expected serial interval~\citep{Banholzer2020} (i.e. time between successive cases in a chain of transmission), which is estimated at about 4-7 days~\citep{Gatto2020}. % Table 1

However, a significant difference between the beginning of the intervention and the effective change in transmission rates can invalidate assessments that assume a serial interval of 4-7 days and neglect the delayed or advanced population response to the NPI.
Such a case is illustrated in \autoref{fig:fig4} using data and parameters from Italy.
Here, a lockdown is officially ordered on Mar~10 ($\tau^*$, but its delayed effect on the transmission dynamics starts on Mar~15 ($\hat{\tau}$). If health officials assume the dynamics to immediately change at $\tau^*$, they will expect the number of cases to follow the dashed red line. However, the number of cases will actually follow the black line, leading to a significant different ($\Delta$) between the projections and the realization.


%%% Fig 4 %%%
\begin{figure}[h]
    \centering
	\includegraphics[width=0.5\textwidth]{../figures/Fig4.pdf}
    \caption{
    \textbf{Delayed effective start of NPI leads to under-estimation of daily confirmed cases.}
    Real number of daily cases in Italy in black (markers: data, line: time moving average). Model predictions, assuming a 50\% decrease in transmission rate after the NPI starts, are shown as colored lines with 95\% confidence intervals. 
    Shaded box illustrates a serial interval of seven days.
    \textbf{(A)} Using the official date $\tau^*$ for the start of the NPI,  the model under-estimates the number of cases seven days after the start of the NPI.
    \textbf{(B)} Using the effective date $\hat{\tau}$ for the start of the NPI,  the model correctly estimates the number of cases seven days after the start of the NPI.
    Here, model parameters are MAP estimates for Italy (\hl{TABLE}) but with $\lambda=0.5$ and $\alpha_1=\alpha_2$.
    } 
    \label{fig:fig4}
\end{figure}



\pagebreak
% Discussion
\section*{Discussion}

We have estimated the effective start date of NPIs in several geographical regions using an SEIR epidemiological model and an MCMC parameter estimation framework.
We find examples of both advanced and delayed response to NPIs (\autoref{fig:fig1}).

For example, in Italy and Wuhan, China, the effective start of the lockdowns seems to have occurred 3-5 after the official date (\autoref{fig:fig2}). This could be explained by low compliance. In Italy, for example, a leak about the intent to lockdown Northern provinces results in people leaving those provinces~\citep{Gatto2020}. However, delayed effect of NPIs could also be due to the time required by both the government and the citizens to organize for a lockdown. 
 
In contrast, in most investigated countries, such as Spain and France, transmission rates seem to have been reduced even before official lockdowns were implemented (\autoref{fig:fig3}).
This advanced response is possibly due to adoption of social distancing and similar behavioral adaptations in parts of the population, maybe in response increased risk perception due to domestic or international COVID\=/19-related reports.
This finding may also suggest that severe NPIs, such as lockdowns, were unnecessary, and that milder measures that were adopted by the population, possibly due to government recommendations, media coverage, and social networks, could have been sufficient for epidemic control.
\hl{check if this is true} Indeed, the evidence supports a change in transmission dynamics (i.e. a model with $\tau$) even for Sweden, in which a lockdown was not implemented, suggesting that lockdowns may not be necessary if other NPIs are adopted early enough during the outbreak~\citep{Banholzer2020} (Sweden banned public events on Mar~12, encouraged social distancing on Mar~16, and closed schools on Mar~18~\citep{Flaxman2020}.)

Attempts to asses the effect of NPIs~\citep{Flaxman2020,Banholzer2020} generally assume a 7 day delay between the implementation of the intervention and the observable change in dynamics, due to the characteristic serial interval of COVID\=/19~\citep{Gatto2020}.
However, the delays and advances we have estimated can confuse these assessments and lead to wrong conclusions about the effects of NPIs (\autoref{fig:fig4}). 


We have found that the evidence supports a model in which the parameters change at a specific time point $\tau$ over a model without such a change-point. It may be interesting to investigate if the evidence favors a model with \emph{two} change-points, rather than one. 
Two such change-points could reflect escalating NPIs (e.g. school closures followed by lockdowns), a mix of NPIs and changes in weather, a mix of domestic and international effects on risk perception, or other similar factors.

As several countries (e.g. Austria, Israel) begin to relieve lockdowns and ease restrictions, we expect similar delays and advances to occur: in some countries people will begin to behave as if restrictions were eased even before the official date, and in some countries people will continue to self-restrict even after restrictions are officially removed.

\paragraph*{Conclusions.}
We have estimated the effective start date of NPIs and found that they often differ from the official dates.
Our results emphasize the complex interaction between personal, regional, and global determinants of behavioral. Thus, our results highlight the need to further study variability in compliance and behavior over both time and space. This can be accomplished both by surveying differences in compliance within and between populations~\citep{Atchison2020}, and by incorporating specific behavioral models into epidemiological models~\citep{Fenichela2011,Arthur2020}.

%\pagebreak
% Acknowledgements
{\small
\section*{Acknowledgements}
%We thank XXX for discussions and comments.
This work was supported in part by the Israel Science Foundation 552/19 and 1399/17.
}


\pagebreak
\begin{multicols}{2}[\section*{References}]
\nolinenumbers
\bibliographystyle{agsm}
%\bibliography{/Users/yoavram/Documents/library}
\bibliography{ms}
\end{multicols}

\pagebreak
\section*{Supplementary Material}
\beginsupplement % https://support.authorea.com/en-us/article/how-to-create-an-appendix-section-or-supplementary-information-1g25i5a/


%%% Fig Fr-vs-Es %%%
\begin{figure}[h]
    \centering
	\includegraphics[width=\textwidth]{../figures/Fig-fr-vs-es.pdf}
    \caption{
    \textbf{COVID\=/19 confirmed cases in France and Spain.}
    Number of cases proportional to population size (as of 2018). 
    Vertical line shows Mar~8, the effective start of NPIs $\hat{\tau}$ in both countries.
    } 
    \label{fig:fig-fr-vs-es}
\end{figure}




\end{document}  