\documentclass[12pt]{extarticle}
\usepackage{geometry}
\geometry{
a4paper,
total={170mm,257mm},
left=20mm,
top=20mm,
headheight=12pt
}

%\usepackage[parfill]{parskip} % Activate to begin paragraphs with an empty line rather than an indent
\usepackage{graphicx} % Use pdf, png, jpg, or eps§ with pdflatex; use eps in DVI mode
% TeX will automatically convert eps --> pdf in pdflatex		

\usepackage{amssymb,amsmath,amsthm}
\usepackage{commath}
\usepackage{longtable}
\usepackage[hyphens]{url}
\usepackage[dvipsnames]{xcolor}
\usepackage[unicode=true,colorlinks=true,urlcolor=CadetBlue,citecolor=black,linkcolor=black]{hyperref}
\def\equationautorefname~#1\null{Eq.~(#1)\null}
\PassOptionsToPackage{hyphens}{url} % url is loaded by hyperref
\usepackage{authblk}
\usepackage{lipsum}
\usepackage{multicol}
\usepackage{titlesec}	
\usepackage{caption}
\usepackage{enumitem}
\usepackage{soul}
      
%SetFonts
% newtxtext+newtxmath
\usepackage{newtxtext} %loads helv for ss, txtt for tt
\usepackage{amsmath}
\usepackage[bigdelims]{newtxmath}
\usepackage[T1]{fontenc}
\usepackage{textcomp}
%SetFonts

% less space before sections 
% \@startsection {NAME}{LEVEL}{INDENT}{BEFORESKIP}{AFTERSKIP}{STYLE} 
%            optional * [ALTHEADING]{HEADING} 
\makeatletter
 \renewcommand\section{\@startsection {section}{1}{\z@}%
     {-2.5ex \@plus -1ex \@minus -.2ex}%
     {1.3ex \@plus.2ex}%
    {\Large\bfseries}}
    
% Species names
%% Meta-Command for defining new species macros
\usepackage{xspace}

\newcommand{\species}[3]{%
  \newcommand{#1}{\gdef#1{\textit{#3}\xspace}\textit{#2}\xspace}}
  
\species{\yeast}{Saccharomyces cerevisiae}{S.~cerevisiae}
\species{\calbicans}{Candida albicans}{C.~albicans}
\species{\cneoformans}{Cryptococcus neoformans}{C.~neoformans}

% line numbers
 \usepackage[displaymath, mathlines]{lineno}
 \renewcommand\linenumberfont{\normalfont\small\sffamily}
\linenumbers
% \modulolinenumbers[2]

% Yoav & Lee commands
\newcommand*{\tr}{^\intercal}
\let\vec\mathbf
\newcommand{\matrx}[1]{{\left[ \stackrel{}{#1}\right]}}
\newcommand{\diag}[1]{\mbox{diag}\matrx{#1}}
\newcommand{\goesto}{\rightarrow}
\newcommand{\dspfrac}[2]{\frac{\displaystyle #1}{\displaystyle #2} }
\newtheorem{theorem}{Theorem}
\newtheorem{corollary}{Corollary}
\newtheorem{lemma}{Lemma}
\newtheorem{remark}{Remark}
\newtheorem{result}{Result}
\renewcommand\qedsymbol{} % no square at end of proof
\newcommand{\cl}{\mathbf{L}}
\newcommand{\cj}{\mathbf{J}}
\newcommand{\ci}{I}

% NatBib
\usepackage{natbib}
%\usepackage[round,colon,authoryear]{natbib}
\renewcommand{\bibsection}{}
\renewcommand{\bibfont}{\small}

% unbreakable dashes
\usepackage[shortcuts]{extdash}

% Title page
% Chromosomal duplication is a transient evolutionary solution to stress
\title{TITLE}

% Authors
\renewcommand\Affilfont{\small}

\author[a]{Ilia Kohanovski}
\author[b,c]{Uri Obolski}
\author[a,*]{Yoav Ram}

\affil[a]{School of Computer Science, Interdisciplinary Center Herzliya, Herzliya 4610101, Israel}
\affil[b]{School of Public Health, Tel Aviv University, Tel Aviv 6997801, Israel}{
\affil[c]{Porter School of the Environment and Earth Sciences, Tel Aviv University, Tel Aviv 6997801, Israel}
\affil[*]{Corresponding author: yoav@yoavram.com}

% Document
\begin{document}
\maketitle

% Abstract
\begin{abstract}
\lipsum[1-1]
\end{abstract}

\pagebreak
% Introduction
\section*{Introduction}

\lipsum[2-4]

\pagebreak
% Models and Methods
\section*{Models and Methods}

%%% Data %%%%
\paragraph*{Data.} 
We use daily confirmed case data $\vec{X}=(X_1, \ldots, X_T)$ from several different countries. These incidence data summarize the number of individuals $X_t$ tested positive for SARS\=/CoV\=/2 RNA (using RT-qPCR) at each day $t$.
Data was retrieved from \hl{REFS} for the following regions: Wuhan, China; Austria; \hl{$\ldots$}. % maybe use a table?

%%% SEIR model %%%%
\paragraph*{SEIR model.}
We model SARS\=/CoV\=/2 infection dynamics by following the number of susceptible $S$, exposed $E$, reported infected $I_r$, and unreported infected $I_u$ individuals in a population of size $N$.
This model distinguishes between reported and unreported infected individuals: the reported infected are those that have enough symptoms to eventually be tested and thus appear in daily case reports, to which we fit the model.

Susceptible ($S$) individuals become exposed due to contact with reported or unreported infected individuals ($I_r$ or $I_u$) at a rate $\beta_t$ or $\mu \beta_t$.
The parameter $0 < \mu < 1$ represents the decreased transmission rate from unreported infected individuals, who are often subclinical or even asymptomatic.
The transmission rate $\beta_t \ge 0$ may change over  time $t$ due to behavioral changes of both susceptible and infected individuals.
Exposed individuals, after an average incubation period of $Z$ days, become reported infected with probability $\alpha_t$ or unreported infected with probability $(1-\alpha_t)$.
The reporting rate $0 < \alpha_t < 1$ may also change over time due to changes in human behavior.
Infected individuals remain infectious for an average period of $D$ days, after which they either recover, or becomes ill enough to be quarantined.
They therefore no longer infect other individuals, and therefore the model does not track their frequency.
The model is described by the following equations:

\begin{equation} \label{eq:model}
\begin{aligned}
\frac{dS}{dt} & = -\beta_t S \frac{I_p}{N} - \mu \beta_t S \frac{I_s}{N} \\
\frac{dE}{dt} & = \beta_t S \frac{I_p}{N} + \mu \beta_t S \frac{I_s}{N}  - \frac{E}{Z} \\
\frac{dI_r}{dt} & = \alpha_t \frac{E}{Z} - \frac{I_r}{D} \\
\frac{dI_u}{dt} & = (1-\alpha_t) \frac{E}{Z} - \frac{I_r}{D} .
\end{aligned}
\end{equation}

This model is inspired by \citet{Li2020} and \citet{Pei2020}, who used a similar model with multiple regions and constant transmission $\beta$ and reporting rate $\alpha$ to infer COVID-19 dynamics in China and the continental US, respectively.

%%% Likelihood %%%
\paragraph*{Likelihood function.}
The \emph{expected} number of new reported infected individuals on day $t$ is 
% UO:   "The expected number of new reported infected individuals on day t is Yt = αtE(t)/Z " I think that this is the new reported cases at time t. Isn't the number of infected something like the integral of the between t-1 and t?
$$
Y_t=\alpha_t E(t)/Z.
$$
We define $\tilde{Y}_t$ to be the cumulative expected number of reported infected individuals up to day $t$,
$$
\tilde{Y}_t = \sum_{i=1}^{t}{Y_i}
$$
As mentioned above, $X_t$ is the number of confirmed cases in day $t$. Then, 
$$
\tilde{X}_t=\sum_{i=1}^{t}X_i
$$
is the cumulative number of confirmed cases until day $t$ (with $X_0=0$).
We assume that reported infected individuals are confirmed and therefore observed in the daily case report of day $t$ with probability $p_t$. Of course, an individual can only be observed once. Note that $p_t$ may change over time, but here $t$ is a specific date, and not the time elapsed since the individual was infected.
Therefore, we assume that the number of confirmed cases in day $t$ is binomially distributed,
$$
X_t \sim \mathit{Bin}\big(n_t, p_t\big),
$$
where $n_t$ is the \emph{realized} number of reported infected individuals yet to appear in daily reports by day~$t$.
Given $\tilde{X}_{t-1}$, we assume $n_t$ is Poisson distributed,
$$
\big(n_t \mid \tilde{X}_{t-1}\big) \sim \mathit{Poi}\Big( \tilde{Y}_t - \tilde{X}_{t-1} \Big), \quad n_1 \sim \mathit{Poi}(Y_1).
$$ 
Therefore, $\big(X_t \mid \tilde{X}_{t-1} \big)$ is a binomial conditioned on a Poisson, which reduces to a Poisson with
\begin{equation} \label{eq:P(Xt|Xt-1)}
\big(X_t \mid \tilde{X}_{t-1} \big) \sim \mathit{Poi}\Big( \big( \tilde{Y}_t - \tilde{X}_{t-1} \big)p_t \Big), \quad X_1 \sim \mathit{Poi}(Y_1 p_1).
\end{equation}
Therefore, for given vector $\theta$ of model parameters 
$$
\theta=\Big(Z, D, \mu, \{\beta_t\}, \{\alpha_t\}, \{p_t\}, S(0), E(0), I_r(0), I_u(0)\Big),
$$ % TODO should p_t be a constant rather than a parameter? Same for S(0) and I_r(0).
which also includes the initial conditions (state at $t=0$), it is possible to compute the expected number of exposed $\{E(t)\}_{t=1}^{T}$ and number of new infections $\{Y_t\}_{t=1}^{T}$ for each day. Then, since $\tilde{X}_{t-1}$ is a function of $X_1, \ldots, X_{t-1}$, we can use \autoref{eq:P(Xt|Xt-1)} to write the probability of the confirmed case data $\vec{X} = (X_1, \ldots, X_T)$ as 
\begin{equation} \label{eq:likelihood}
\mathbb{L}(\theta \mid \vec{X}) = P(\vec{X} \mid \theta) = P(X_1 \mid \theta) P(X_2 \mid X_1, \theta) \cdots P(X_T \mid X_1, \ldots X_{T-1}, \theta).
\end{equation}
This defines our \emph{likelihood function} for the parameter vector $\theta$ given the data $\vec{X}$.

%%% NPI %%%
\paragraph*{NPI model.}
To model non-pharmaceutical interventions (NPIs), we set the start of the NPIs to day $\tau$ and define
$$
\beta_t = \begin{cases} 
  \beta, & t < \tau \\ % TODO make sure < and not \le, and the next line \ge and not >
  \beta \lambda, & t \ge \tau
\end{cases},
\quad
\alpha_t = \begin{cases} 
  \alpha_1, & t < \tau \\ % TODO make sure < and not \le, and the next line \ge and not >
  \alpha_2, & t \ge \tau
\end{cases},
\quad
p_t = \begin{cases} 
  1/9, & t < \tau \\ % TODO make sure < and not \le, and the next line \ge and not >
  1/6, & t \ge \tau
\end{cases},
$$
where $0 < \lambda < 1$.
The values for $p_t$ follow \citet{Li2020}, who estimated the average time between infection and reporting in Wuhan, China, at 9 days before the start of NPIs (Jan 23, 2020) and 6 days after start of NPIs.
The parameter $\tau$ is then part of the parameter vector $\theta$.

%%% Model fitting %%%
\paragraph*{Model fitting.}
To fit our model (\autoref{eq:model}) to the data $\vec{X}$ and estimate the model parameters $\theta$, we apply a Bayesian inference approach.
We define the following flat priors on the model parameters $P(\theta)$:
\begin{equation} \label{eq:priors}
\begin{aligned} % TODO verify before submitting
Z & \sim \mathit{Uniform}(2, 5) \\
D & \sim \mathit{Uniform}(2, 5) \\
\mu & \sim \mathit{Uniform}(0.2, 1) \\
\beta & \sim \mathit{Uniform}(0.8, 1.5) \\
\lambda & \sim \mathit{Uniform}(0, 1) \\
\alpha_1, \alpha_2 & \sim \mathit{Uniform}(0.02, 1)\\
E(0) & \sim \mathit{Uniform}(0, 3000) \\
I_u(0) & \sim \mathit{Uniform}(0, 3000) \\
\tau &\sim \mathit{Uniform}(1, T-1),
\end{aligned}
\end{equation}
where $T$ is the number of days in the data $\vec{X}$.
Most priors follow \citet{Li2020}, except $\lambda$, which is used to enforce that the transmission rates are lower after the start of the NPIs ($\beta_{t\ge \tau} < \beta_{t < \tau}$).
The posterior distribution on the model parameters $P(\theta \mid \vec{X})$ is then estimated using an affine-
invariant ensemble sampler for Markov chain Monte Carlo (MCMC) implemented in the \texttt{emcee} Python package~\citep{Foreman-Mackey2013}.

\pagebreak
% Results
\section*{Results}

\pagebreak
% Discussion
\section*{Discussion}

\lipsum[4-6]

As several countries (e.g. Austria, Israel) begin to relieve lockdowns and ease restrictions, we expect similar delays and advances to occur: in some countries people will begin to behave as if restrictions were eased before the official date, and in some countries people will continue to self-restrict even after restrictions are officially removed.
Such delays and advances could confuse analyses and lead to wrong conclusions about the effects of NPI removals.

\pagebreak
% Acknowledgements
{\small
\section*{Acknowledgements}
%We thank XXX for discussions and comments.
This work was supported in part by the Israel Science Foundation 552/19 (YR) and XXX/XX (Alon Rosen)
}

\pagebreak
\begin{multicols}{2}[\section*{References}]
\nolinenumbers
\bibliographystyle{agsm}
\bibliography{/Users/yoavram/Documents/library}
%\bibliography{ms}
\end{multicols}

\end{document}  