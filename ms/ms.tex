\documentclass[12pt]{extarticle}
\usepackage{geometry}
\geometry{
a4paper,
total={170mm,257mm},
left=20mm,
top=20mm,
headheight=12pt
}

%\usepackage[parfill]{parskip} % Activate to begin paragraphs with an empty line rather than an indent
\usepackage{graphicx} % Use pdf, png, jpg, or eps§ with pdflatex; use eps in DVI mode
% TeX will automatically convert eps --> pdf in pdflatex		

\usepackage{amssymb,amsmath,amsthm}
\usepackage{commath}
\usepackage{longtable}
\usepackage[hyphens]{url}
\usepackage[dvipsnames]{xcolor}
\usepackage[unicode=true,colorlinks=true,urlcolor=CadetBlue,citecolor=black,linkcolor=black]{hyperref}
\def\equationautorefname~#1\null{Eq.~(#1)\null}
\PassOptionsToPackage{hyphens}{url} % url is loaded by hyperref
\usepackage{authblk}
\usepackage{lipsum}
\usepackage{multicol}
\usepackage{titlesec}	
\usepackage{caption}
\usepackage{enumitem}
\usepackage{soul}
      
%SetFonts
% newtxtext+newtxmath
\usepackage{newtxtext} %loads helv for ss, txtt for tt
\usepackage{amsmath}
\usepackage[bigdelims]{newtxmath}
\usepackage[T1]{fontenc}
\usepackage{textcomp}
%SetFonts

% less space before sections 
% \@startsection {NAME}{LEVEL}{INDENT}{BEFORESKIP}{AFTERSKIP}{STYLE} 
%            optional * [ALTHEADING]{HEADING} 
\makeatletter
 \renewcommand\section{\@startsection {section}{1}{\z@}%
     {-2.5ex \@plus -1ex \@minus -.2ex}%
     {1.3ex \@plus.2ex}%
    {\Large\bfseries}}
    
% Species names
%% Meta-Command for defining new species macros
\usepackage{xspace}

\newcommand{\species}[3]{%
  \newcommand{#1}{\gdef#1{\textit{#3}\xspace}\textit{#2}\xspace}}
  
\species{\yeast}{Saccharomyces cerevisiae}{S.~cerevisiae}
\species{\calbicans}{Candida albicans}{C.~albicans}
\species{\cneoformans}{Cryptococcus neoformans}{C.~neoformans}

% line numbers
 \usepackage[displaymath, mathlines]{lineno}
 \renewcommand\linenumberfont{\normalfont\small\sffamily}
\linenumbers
% \modulolinenumbers[2]

% Yoav & Lee commands
\newcommand*{\tr}{^\intercal}
\let\vec\mathbf
\newcommand{\matrx}[1]{{\left[ \stackrel{}{#1}\right]}}
\newcommand{\diag}[1]{\mbox{diag}\matrx{#1}}
\newcommand{\goesto}{\rightarrow}
\newcommand{\dspfrac}[2]{\frac{\displaystyle #1}{\displaystyle #2} }
\newtheorem{theorem}{Theorem}
\newtheorem{corollary}{Corollary}
\newtheorem{lemma}{Lemma}
\newtheorem{remark}{Remark}
\newtheorem{result}{Result}
\renewcommand\qedsymbol{} % no square at end of proof
\newcommand{\cl}{\mathbf{L}}
\newcommand{\cj}{\mathbf{J}}
\newcommand{\ci}{I}

% NatBib
\usepackage[super,comma,sort&compress]{natbib}
%\usepackage[round,colon,authoryear]{natbib}
\renewcommand{\bibsection}{}
\renewcommand{\bibfont}{\small}

% unbreakable dashes
\usepackage[shortcuts]{extdash}

% Title page
% Chromosomal duplication is a transient evolutionary solution to stress
\title{TITLE}

% Authors
\renewcommand\Affilfont{\small}

\author[a]{Ilia Kohanovski}
\author[b,c]{Uri Obolski}
\author[a,*]{Yoav Ram}

\affil[a]{School of Computer Science, Interdisciplinary Center Herzliya, Herzliya 4610101, Israel}
\affil[b]{School of Public Health, Tel Aviv University, Tel Aviv 6997801, Israel}{
\affil[c]{Porter School of the Environment and Earth Sciences, Tel Aviv University, Tel Aviv 6997801, Israel}
\affil[*]{Corresponding author: yoav@yoavram.com}

% Document
\begin{document}
\maketitle

% Abstract
\begin{abstract}
\lipsum[1-1]
\end{abstract}

\pagebreak
% Introduction
\section*{Introduction}

\lipsum[2-4]

\pagebreak
% Models and Methods
\section*{Models and Methods}

%%% Data %%%%
\paragraph*{Data.} 
We use daily confirmed case data $\vec{X}=(X_1, \ldots, X_T)$ from several different countries. These incidence data summarize the number of individuals $X_t$ tested positive for SARS\=/CoV\=/2 RNA (using RT-qPCR) at each day $t$.
Data was retrieved from \hl{REFS} for the following regions: Wuhan, China; Austria; \hl{$\ldots$}. % maybe use a table?

%%% SEIR model %%%%
\paragraph*{SEIR model.}
We model SARS\=/CoV\=/2 infection dynamics by following the number of susceptible $S$, exposed $E$, reported infected $I_r$, and unreported infected $I_u$ individuals in a population of size $N$.
This model distinguishes between reported and unreported infected individuals: the reported infected are those that have enough symptoms to eventually be tested and thus appear in daily case reports, to which we fit the model.

Susceptible ($S$) individuals become exposed due to contact with reported or unreported infected individuals ($I_r$ or $I_u$) at a rate $\beta_t$ or $\mu \beta_t$.
The parameter $0 < \mu < 1$ represents the decreased transmission rate from unreported infected individuals, who are often subclinical or even asymptomatic.
The transmission rate $\beta_t$ may change over  time $t$ due to behavioral changes of both susceptible and infected individuals.
Exposed individuals, after an average incubation period of $Z$ days, become reported infected with probability $\alpha_t$ or unreported infected with probability $(1-\alpha_t)$.
The reporting rate $\alpha_t$ may also change over time due to changes in human behavior.
Infected individuals remain infectious for an average period of $D$ days, after which they either recover, or becomes ill enough to be quarantined.
They therefore no longer infect other individuals, and therefore the model does not track their frequency.
The model is described by the following equations:

\begin{equation}
\begin{aligned}
\frac{dS}{dt} & = -\beta_t S \frac{I_p}{N} - \mu \beta_t S \frac{I_s}{N} \\
\frac{dE}{dt} & = \beta_t S \frac{I_p}{N} + \mu \beta_t S \frac{I_s}{N}  - \frac{E}{Z} \\
\frac{dI_r}{dt} & = \alpha_t \frac{E}{Z} - \frac{I_r}{D} \\
\frac{dI_u}{dt} & = (1-\alpha_t) \frac{E}{Z} - \frac{I_r}{D} .
\end{aligned}
\end{equation}

This model is inspired by \citet{Li2020} and \citet{Pei2020}, who used a similar model with multiple regions and constant transmission $\beta$ and reporting rate $\alpha$ to infer COVID-19 dynamics in China and the continental US, respectively.

%%% Likelihood %%%
\paragraph*{Likelihood function.}
The \emph{expected} number of new reported infected individuals on day $t$ is 
$$
Y_t=\alpha_t E(t)/Z.
$$
We define $\tilde{Y}_t$ to be the cumulative expected number of reported infected individuals up to day $t$,
$$
\tilde{Y}_t = \sum_{i=1}^{t}{Y_i}
$$
As mentioned above, $X_t$ is the number of confirmed cases in day $t$. Then, 
$$
\tilde{X}_t=\sum_{i=1}^{t}X_i
$$
is the cumulative number of confirmed cases until day $t$ (with $X_0=0$).
We assume that reported infected individuals yet to be confirmed, i.e. individuals in $\tilde{Y}_t$, are confirmed and therefore appear in the daily case report of day $t$ with probability $p_t$, which may change over time (note that $t$ is a specific date, and not the elapsed time since infection).
Therefore, we assume that the number of confirmed cases in day $t$ is binomially distributed,
$$
X_t \sim \mathit{Bin}\big(n_t, p_t\big),
$$
where $n_t$ is the \emph{realized} number of reported infected individuals yet to appear in daily reports by day~$t$.
Given $\tilde{X}_{t-1}$, we assume $n_t$ is Poisson distributed,
$$
\big(n_t \mid \tilde{X}_{t-1}\big) \sim \mathit{Poi}\Big( \tilde{Y}_t - \tilde{X}_{t-1} \Big), \quad n_1 \sim \mathit{Poi}(Y_1).
$$ 
Therefore, $\big(X_t \mid \tilde{X}_{t-1} \big)$ is a binomial conditioned on a Poisson, which reduces to a Poisson with
\begin{equation} \label{eq:P(Xt|Xt-1)}
\big(X_t \mid \tilde{X}_{t-1} \big) \sim \mathit{Poi}\Big( \big( \tilde{Y}_t - \tilde{X}_{t-1} \big)p_t \Big), \quad X_1 \sim \mathit{Poi}(Y_1 p_1).
\end{equation}
Therefore, for given vector $\theta$ of model parameters 
$$
\theta=\Big(Z, D, \mu, \{\beta_t\}, \{\alpha_t\}, \{p_t\}, S(0), E(0), I_r(0), I_u(0)\Big),
$$
which also includes the initial conditions (state at $t=0$), it is possible to compute the expected number of exposed $\{E(t)\}_{t=1}^{T}$ and number of new infections $\{Y_t\}_{t=1}^{T}$ for each day. Then, since $\tilde{X}_{t-1}$ is a function of $X_1, \ldots, X_{t-1}$, we can use \autoref{eq:P(Xt|Xt-1)} to write the probability of the confirmed case data $\vec{X} = (X_1, \ldots, X_T)$ as 
\begin{equation} \label{eq:likelihood}
\mathbb{L}(\theta \mid \vec{X}) = P(\vec{X} \mid \theta) = P(X_1 \mid \theta) P(X_2 \mid X_1, \theta) \cdots P(X_T \mid X_1, \ldots X_{T-1}, \theta).
\end{equation}
This defines our \emph{likelihood function} for the parameter vector $\theta$ given the data $\vec{X}$.

\pagebreak
% Results
\section*{Results}

\pagebreak
% Discussion
\section*{Discussion}

\lipsum[4-6]

\pagebreak
% Acknowledgements
{\small
\section*{Acknowledgements}
%We thank XXX for discussions and comments.
This work was supported in part by the Israel Science Foundation 552/19 (YR) and XXX/XX (Alon Rosen)
}

\pagebreak
\begin{multicols}{2}[\section*{References}]
\nolinenumbers
\bibliographystyle{agsm}
%\bibliography{/Users/yoavram/Documents/library}
\bibliography{ms}
\end{multicols}

\end{document}  